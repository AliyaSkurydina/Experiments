\documentclass[10pt,pdf, mathserif, hyperref={unicode}]{beamer}

% \documentclass[aspectratio=43]{beamer}
% \documentclass[aspectratio=1610]{beamer}
% \documentclass[aspectratio=169]{beamer}

%\usepackage{lmodern}
\usepackage{multirow}
\usepackage{colortbl}
\usepackage{color,ucs}
\usepackage{xcolor}

% подключаем кириллицу 
\usepackage[T2A]{fontenc}
\usepackage[utf8]{inputenc}

% отключить клавиши навигации
\setbeamertemplate{navigation symbols}{}

% тема оформления
\usetheme{Berlin}

% цветовая схема
\usecolortheme{seahorse}
% сглаживаем углы
\useinnertheme{rounded}

\useoutertheme{shadow}
\linespread{1.4} %1.3 для полуторного
\graphicspath{{fig/}}
\definecolor{Green}{rgb}{0,1,0}


\begin{document}	


\begin{frame}{Ведущая организация: 
		Казанский (Приволжский) федеральный университет, г. Казань}
Подписан:  доцент кафедры анализа данных и исследования операций КФУ,  к. ф.-м. н., В. В. Бандеров,\\
профессор кафедры вычислительной математики КФУ,  д. ф.-м. н., чл.-корр. Академии наук Републики Татарстан, И. Б. Бадриев,\\
зав. кафедрой  вычислительной математики КФУ,  д. ф.-м. н., профессор О. А. Задворнов.\\
Утвержден: Проректор по научной деятельности, д. г.-м. н., профессор Д. К. Нургалиев.

\end{frame}
\begin{frame}{Замечание 1}
	\textit {При аппроксимации интегрального оператора задач гравиметрии и магнитометрии по квадратурным формулам не учитывается погрешность}.
	\vskip 5mm
	Ответ:\\
	Данный вопрос в рамках диссертационной работы не исследовался, но предполагается, что погрешность дискретизации намного меньше погрешности, с которой задана правая часть операторного уравнения $$A(u)+\alpha(u-u^0)=f_\delta,\quad \|f-f_\delta\|\leqslant\delta.$$
\end{frame}
\begin{frame}{Замечание 2}
	\textit {При решении задач гравиметрии и магнитометрии с шумом стоило бы описать, откуда берется шум}.
	\vskip 5mm
	Ответ:\\
	Согласна. Источников возникновения шумов может быть несколько: вкрапления пород с различной плотностью или намагниченностью в слоях среды, ошибки измерительных приборов (гравиметров или магнитометров), погрешности процедур предварительной обработки (например, учет рельефа).
\end{frame}

\begin{frame}{Замечание 3}
	\textit {В тексте диссертации стоило бы подробнее пояснить о вычислениях <<на лету>> на видеокартах}.
	\vskip 5mm
	Ответ:\\
	Согласна. В тексте диссертации на стр. 92 приводится объяснение: «необходимый элемент матрицы вычисляется в момент умножения его на элемент вектора». Это означает, что в программном коде в том же цикле, где производится матрично-векторная операция, элемент матрицы вычисляется по формуле 
	$$ [A'(u^0)]h=\iint_{D} \frac{u^0(x',y')h(x',y')}{[(x-x')^2+(y-y')^2+(u^0(x',y'))^2]^{3/2}}dx'dy',$$ а затем умножается на элемент вектора, который хранится в памяти. Хранения самой матрицы в памяти не производится. Результат матрично-векторного умножения сохраняется в память.
\end{frame}

\begin{frame}{Замечание 4}
	\textit {В тексте присутствуют стилистические ошибки и опечатки. Например, на странице 56 вместо <<зна>> следовало бы указать <<на>>; на странице 58 в определении функции $f$ должен стоять знак +, а не минус}.
	\vskip 5mm
	Ответ:\\
	С замечанием согласна.
\end{frame}

\begin{frame}{Замечание 5}
	\textit {На страницах 14, 42, 44 желательно было бы указать примеры задач с немонотонными операторами}.
	\vskip 5mm
	Ответ:\\
	Нелинейный интегральный оператор гравиметрии, а также магнитометрии не является монотонным. Примеры задач с немонотонными операторами подробно рассматриваются в разделе 4 главы 2. 
\end{frame}

\begin{frame}{Замечание 6}
	\textit {В основных результатах есть утверждение об обобщении результатов для нелинейных аналогов $\alpha$-процессов на конечномерный случай – в тексте диссертации об этом не говорится}.
	\vskip 5mm
	Ответ:\\
	На стр. 43 диссертации рассматривается конечномерный случай, в главе 2 в конечномерном случае для немонотонного оператора доказаны аналогичные теоремы, что и в главе 1. Это и есть обобщение.
\end{frame}

\begin{frame}{Замечания официального оппонента}
	Танана Виталий Павлович, д. ф.-м. н., профессор, главный научный сотрудник кафедры системного
	программирования Южно-Уральского государственного университета (национального исследователь-ского университета), г. Челябинск
	
\end{frame}

\begin{frame}{Замечание 1}
	\textit{В  разделе 1.1  главы 1 не поясняется, почему  в методах Ньютона и нелинейных аналогах $\alpha$-процессов используются разные параметры регуляризации $\alpha$ и $\bar{\alpha}$}?
	\vskip 5mm
	Ответ:\\
	Использование разных параметров регуляризации позволяют доказать сильную фейеровость метода Ньютона и аналогов $\alpha$-процессов в главах 1,2 и обосновать двухэтапный метод. На параметр $\bar{\alpha}$ накладывается ограничение снизу. Кроме того, изменение параметра $\bar{\alpha}$  без изменения $\alpha$ позволяет остаться в условиях теорем главы 2, когда спектр матрицы производной оператора A содержит набор малых по абсолютной величине отрицательных собственных значений (замечание 2.3.)
\end{frame}

\begin{frame}{Замечание 2}
	\textit{В работе имеются опечатки. На стр. 23 в строке 1 пропущена запятая перед союзом <<и>>. На стр. 80 в строке 5 после слова <<раздела>> пропущено слово <<сред>>. На стр. 97 опечатка в строке 7 в слове <<сравнения>> (пропущена буква). В таблице 3.1 дробную и целую части следовало бы отделить единообразно - либо точкой, либо запятой}.
	\vskip 5mm
	Ответ:\\
	С замечанием согласна.
\end{frame}

\begin{frame}{Замечания официального оппонента}
	Ягола Анатолий Григорьевич, д. ф.-м. н., профессор кафедры математики физического факультета Московского государственного университета, г. Москва
	
\end{frame}

\begin{frame}{Замечание 1}
	\scriptsize
	\textit{На стр. 37 рассматривается задача Коши для обыкновенного дифференциального ypавнения первого порядка $$	\frac{dy}{dt}=x(t)y(t), \quad y(0)=c_0.$$ При этом утверждается, что входящие в уравнение функции $x(t)$, $y(t)$ принадлежат пространству $L^2[0,1]$ Как в этом случае определяется решение, диссертант не определяет.}
	\vskip 5mm
	Ответ:\\
	При условии, что $x(t)\in L^2[0,1]$, ДУ имеет дифференцируемое почти всюду решение $y(t)$, поэтому решение ДУ можно понимать в обыкновенном смысле. Условие  принадлежности $x(t), y(t)$ пространству $L^2[0,1]$ следовало отнести не к ДУ, а к полученному интегральному уравнению, которое необходимо решать на паре гильбертовых пространств.
	$$[F(x)](t)=c_0 e^{\int_{0}^{t}x(\tau)d\tau}=y(t),$$
	$F$ действует из $L^2[0,1]$ в $L^2[0,1]$.
	
	Автор признает, что эта небрежность и породила вопрос Анатолия Григорьевича.
\end{frame}

\begin{frame}{Замечание 2}
	\textit{В тексте присутствуют стилистические ошибки и опечатки. На стр. 28 в строке 6 пропущена запятая перед союзом <<и>>, после слова <<при>> стоило написать <<$\gamma=\gamma^{\textnormal opt}$>>. На стр. 29 в строке 10 стоило написать  <<аналоги $\alpha$-процессов>> вместо <<$\alpha$-процессы>>. На стр. 46 в строке 11 запятая после <<в соотношении (1.30)>> --- лишняя, в строке 14 пропущена запятая после слова <<методу>>. На стр. 47 в строке 8 пропущены запятые перед словами <<при>> и <<получаем>>. На стр. 84 в строке 3 пропущена запятая после слов <<гравитационного поля>>. На стр. 92 в строке 19 слово <<были>> --- лишнее}.
	\vskip 5mm
	Ответ:\\
	С замечанием согласна.
\end{frame}

\begin{frame}{Замечание 3}
	\scriptsize\textit{Допущены неточности при написании обзора литературы. Так, ссылки [113] (стр. 6) не существует:
		Тихонов А. Н, Арсенин В. Я. Приближенное решение операторных уравнений. - Москва: Наука, 1986.
		Работа [92] была одной из первых, но далеко не единственной из серии работ, посвященных регуляризуемости некорректных задач. Обзор методов регyляризации при условии, что решение операторного уравнения является функцией ограниченной вариации (стр. 8), не содержит ссылок на работы А.С. Леонова получившего наиболее существенные результаты. Зато есть ссылка на статью И.Ф. Дорофеева [61], которая, как выяснилось позднее, содержала принципиальные ошибки. Статья А. В. Гончарского, А.С.Леонова, А.Г.Яголы [58] заодно приписана М.Г. Дмитриеву, В. С. Полещук [60]
		}.
	\vskip 5mm
	Ответ:\\
	Согласна. Допущены опечатки и технические ошибки в списке литературы, книга Тихонова А. Н, Арсенина В. Я. называется <<Методы решения некорректных задач>>, статья А. В. Гончарского, А.С.Леонова, А.Г.Яголы [58] некорректно продублирована.
\end{frame}

\begin{frame}{Замечания на автореферат}
	Прохоров Игорь Васильевич, д. ф.-м. н., зам. директора по научной работе Институт прикладной мате-матики ДВО РАН, г. Владивосток
	\vskip 5mm
	Замечаний нет.
\end{frame}
\begin{frame}
	Эпов Михаил Иванович, д. т. н., академик РАН, директор,\\
	\vskip 5mm
	Глинских Вячеслав Николаевич, д. ф.-м. н., доцент, заведующий лабораторией скважинной геофизики,\\
	\vskip 5mm
	Институт нефтегазовой геологии и геофизики СО РАН, г. Новосибирск 
	
\end{frame}
\begin{frame}{Замечания}
	Наиболее полно раскрыть полученные результаты позволили бы следующие, не освещенные в автореферате, но, возможно отмеченные в диссертации, аспекты:
	
	1.	зависимость времени решения задач от количества используемых вычислительных ядер;
	
	2.	зависимость производительности (быстродействия) алгоритмов от размеров используемой сетки;
	
	3.	предложены модифицированные версии алгоритмов Ньютона и Левенберга -- Марквардта, превосходящие оригинальные версии, но не проводится их сравнение между собой.
	
\end{frame}

\begin{frame}{}
	
	Ответы:
	1, 2. Согласна. Это связано с ограничением объема автореферата. Описание экспериментов с реальными данными и характеристики производительности параллельных алгоритмов есть в тексте диссертации.
	\\
	3. Сравнение покомпонентных методов Ньютона и Левенберга-Марквардта в диссертации не проводится, так как первый метод предназначен для решения задач гравиметрии для модели двухслойной среды, а второй метод – для модели многослойной среды. Покомпонентный метод типа Левенберга -- Марквардта подходит и для решения задачи гравиметрии для модели двухслойной среды, но по быстродействию будет уступать покомпонентному методу типа Ньютона в силу более высокой вычислительной сложности.
\end{frame}

\begin{frame}
	Цымблер Михаил Леонидович, к. ф.-м. н., доцент, нач. отдела интеллектуального анализа данных и виртуализации, 
	\vskip 5mm
	Лаборатория суперкомпьютерного моделирования, Южно-Уральский государственный университет (национальный исследовательский университет), г. Челябинск
\end{frame}

\begin{frame}{Замечание 1}
	\textit{При описании результатов вычислительных экспериментов в главе 3 диссертации автор употребляет неточный термин <<максимальное ускорение>> вместо <<линейное ускорение>> (ускорение может быть сверхлинейным и, строго говоря, не ограничено сверху)}.
	\vskip 5mm
	Ответ:\\
	С замечанием согласна.
\end{frame}

\begin{frame}{Замечание 2}
	\textit{В главе 3 для отображения результатов экспериментов по исследованию ускорения используется табличная форма, и указывается лишь время работы алгоритма на одном ядре и максимально доступном количестве ядер вычислительной системы. Более общепринятым является отображение в виде графика зависимости}.
	\vskip 5mm
	Ответ:\\
	С замечанием согласна. 
\end{frame}

\begin{frame}
	Копысов Сергей Петрович, д. ф.-м. н., главный научный сотрудник,\\
	Новиков Александр Константинович, к. ф.-м. н., старший научный сотрудник,\\
	
	\vskip 5mm
	Институт механики УдмФИЦ УрО РАН, г. Ижевск
	
	\vskip 5mm
	Замечание:\\
	\textit{Утверждение о том, что покомпонентные методы работают в 3 и в 10 раз быстрее методов Ньютона и Левенберга -- Марквардта, соответственно (стр. 16, п. 4), следовало отнести к результатам п.1 и 3}.
	\vskip 5mm
	Ответ:\\
	С замечанием согласна.
\end{frame}

\begin{frame}
	Александр Сергеевич Долгаль, д. ф.-м. н., доцент, главный научный сотрудник,
	
	\vskip 5mm
	Горный институт УрО РАН, г. Пермь
	\vskip 5mm
	Замечание:\\
	\textit{Несколько странно выглядит отсутствие в автореферате защищаемых положений, традиционно формирующих <<скелет>> диссертационной работы}.
	\vskip 5mm
	Ответ:\\
	Вместо положений в диссертации и автореферате на защиту выносятся основные результаты.
\end{frame}

\begin{frame}
	Сергей Иванович Мартыненко,  д. ф.-м. н., научный сотрудник отдела <<Специальные авиационные двигатели и химмотология>>,
	\vskip 5mm
	ФГУП <<ЦИАМ им. П.И. Баранова>>, г. Москва
\end{frame}

\begin{frame}{Замечание 1}
	\scriptsize
	\textit{В автореферате приведены результаты решения задачи гравиметрии при помощи покомпонентных методов типа Ньютона (ПМН) и Левенберга -- Марквардта, однако отсутствует подробный анализ их вычислительной трудоёмкости (сложности). На странице утверждается, что <<Вычислительная сложность ПМН для решения системы n уравнений без учета сложности алгоритма вычисления $A_n(u_n^k)$  составляет O(n)>>. В прикладном аспекте интерес представляет оценка объёма вычислений как функция параметра дискретизации. Из текста автореферата не ясно обладает ли ПМН оптимальной трудоёмкостью? Какова трудоёмкость остальных рассмотренных методов?}.
	\vskip 5mm
	Ответ:\\
	ПМН обладает оптимальной трудоемкостью. Трудоемкость метода Ньютона, нелинейных аналогов $\alpha$-процессов составляет O($n^2$) операций, трудоемкость покомпонентного метода Левенберга -- Марквардта составляет O($n^2$) операций, так как в алгоритмах содержатся операции умножения матрицы на вектор.
\end{frame}

\begin{frame}{Замечание 2}
	\scriptsize
	\textit{Анализ трудоёмкости последовательных алгоритмов важен для построения параллельных алгоритмов. Очевидна бесполезность распараллеливания алгоритмов с трудоёмкостью более O(N lg(N)) арифметических операций, где N есть количество узлов сетки, при наличии оптимального последовательного алгоритма с трудоёмкостью O(N) арифметических операций. Данные о времени исполнения параллельных алгоритмов, приведённые на с. 16, следует дополнить оценками ускорения и эффективности параллелизма}.
	\vskip 5mm
	Ответ:\\
	Методы, у которых трудоемкость выше, чем у покомпонентного метода, исследуются для сравнения параллелизуемости алгоритмов. ПМН не обоснован теоретически, а метод Ньютона, аналоги $\alpha$-процессов обоснованы, поэтому отказаться от их использования нельзя и результаты по их распараллеливанию могут быть востребованы.
\end{frame}

\begin{frame}{Замечание 3}
	\textit{В автореферате практически не описан разработанный комплекс программ}.
	\vskip 5mm
	Ответ:\\
	Описание комплекса программ не приводится из-за ограничения объема автореферата. В диссертации содержится подробное описание комплекса программ. 
\end{frame}

\begin{frame}{Замечания на автореферат}
	Болдырев Юрий Яковлевич, д. т. н., профессор кафедры прикладной математики Санкт-Петербургского политехнического университета Петра Великого, г. Санкт-Петербург.
	\vskip 5mm
	Замечаний нет.
\end{frame}

\end{document}