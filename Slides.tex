\documentclass[10pt,pdf, mathserif, hyperref={unicode}]{beamer}

% \documentclass[aspectratio=43]{beamer}
% \documentclass[aspectratio=1610]{beamer}
% \documentclass[aspectratio=169]{beamer}

%\usepackage{lmodern}
\usepackage{multirow}
\usepackage{colortbl}
\usepackage{color,ucs}
\usepackage{xcolor}

% подключаем кириллицу 
\usepackage[T2A]{fontenc}
\usepackage[utf8]{inputenc}

% отключить клавиши навигации
\setbeamertemplate{navigation symbols}{}

% тема оформления
\usetheme{Berlin}

% цветовая схема
\usecolortheme{seahorse}
% сглаживаем углы
\useinnertheme{rounded}

\useoutertheme{shadow}
\linespread{1.4} %1.3 для полуторного
\graphicspath{{fig/}}
\definecolor{Green}{rgb}{0,1,0}

\title{The title of presentation}   
\author{proft} 
\date{2018} 

\begin{document}	
	\small
	\footnotesize
	
\title[\hspace*{55mm}{\insertpagenumber /\pageref{lastpage}}]{{\small\textbf{Регуляризующие алгоритмы на основе методов 
ньютоновского типа и нелинейных аналогов $\alpha$-процессов}}}
\author[\insertlogo{\em{Алия Фиргатовна Скурыдина}}%
\hspace*{60mm}]{\textbf{\color{blue}Алия Фиргатовна Скурыдина}}
\institute[05.13.18]
{01.01.07 --- Вычислительная математика\\
\vspace*{1cm}
	Научный руководитель: д. ф.-м. н., вед. н. с. ИММ УрО РАН Е. Н. Акимова \\ 
	\vspace*{1cm}}
\frame{\titlepage}

\begin{frame}{Введение}{}
%	Диссертационная работа посвящена построению и исследованию  регуляризованных итерационных методов решения нелинейных некорректных операторных уравнений.
%	
%	\smallskip
	{\color{blue}Актуальность темы.} %Построение итеративно регуляризованных алгоритмов востребовано для решения широкого круга некорректных прикладных задач. Например, решение структурных обратных задач гравиметрии и магнитометрии сводится к решению нелинейных интегральных уравнений первого рода.
	Теория некорректно поставленных задач и методы их решения относятся к важнейшим направлениям исследования современной вычислительной математики, что обусловлено потребностями различных областей естествознания, техники и медицины, где эти проблемы возникают в форме обратных задач.
	
	\smallskip
	%Теорию решения некорректных задач развивали А.~Н.~Тихонов, М.~М.~Лаврентьев, В.~К.~Иванов, А.~Б.~Бакушинский, Б.~Т.~Поляк, А.~В.~Гончарский, В.~В.~Васин, А.~Л.~Агеев, В.~П.~Танана, А.~Г.~Ягола, A.~Neubauer, O.~Scherzer, B.~Kaltenbacher, U.~Tautenhahn и др.
	Основы теории некорректно поставленных задач были заложены в 50--60 годы прошлого века в работах выдающихся российских математиков А.~Н.~Тихонова, В.~К.~Иванова, М.~М.~Лаврентьева. % и дальнейшее ее развитие было продолжено в работах их последователей и учеников.
	%В этих работах исследования относились, главным образом, к линейным уравнениям.% для нелинейных задач были сформулированы базовые принципы регуляризации.

	В работах А.~Б.~Бакушинского предложен принцип итеративной регуляризации метода Ньютона и исследована его сходимость.
	
\end{frame}

\begin{frame}{Введение}
	
Устойчивые методы решения нелинейных некорректных задач 
строились и исследовались в работах А.~Б.~Бакушинского, В.~В.~Васина,  Б.~Т.~Поляка, A.~Neubauer, B.~Kaltenbacher, H.~W.~Engl, А.~Л.~Агеева, А.~В.~Гончарского, С.~И.~Кабанихина, М.~Ю.~Кокурина, А.~С.~Леонова, В.~А.~Морозова, В.~П.~Тананы, А.~Г.~Яголы, M.~Hanke.
%
%%Структурные задачи гравиметрии и магнитометрии --- важный класс нелинейных некорректных задач. При обработке геофизических данных на больших площадях возникает необходимость решать системы нелинейных уравнений большой размерности с использованием параллельных вычислений.
%\vskip 5pt
%%Структурные задачи гравиметрии и магнитометрии --- нелинейные некорректные задачи. 
%%Решение структурных задач гравиметрии и магнитометрии на больших сетках возникает необходимость решать системы нелинейных уравнений большой размерности с использованием параллельных вычислений.
%\vskip 5pt
%\vskip 5pt

В ИММ УрО РАН разработаны и исследованы параллельные алгоритмы на основе регуляризованных методов Ньютона, Левенберга -- Марквардта и процессов градиентного типа (В.~В.~Васин, Е.~Н.~Акимова, Г.~Я.~Пересторонина, Л.~Ю.~Тимерханова, В.~Е.~Мисилов).

В ИГФ УрО РАН разработана оригинальная методика решения обратных задач гравиметрии и магнитометрии с использованием идей регуляризации, построены алгоритмы на основе метода локальных поправок (П.С.~Мартышко,  И.Л.~Пруткин, Н.В.~Федорова, А.Л.~Рублев, И.В.~Ладовский, А.Г.~Цидаев, Д.Д.~Бызов).  

При решении обратных задач гравиметрии и магнитометрии на больших сетках используются параллельные алгоритмы и многопроцессорные вычислители.
\end{frame}

\begin{frame}{Введение}
%	Диссертационная работа посвящена построению и исследованию регуляризованных итерационных методов решения нелинейных некорректных операторных уравнений.
	
	\begin{block}{}
		{\color{blue}Целью} диссертационной работы является построение новых устойчивых и экономичных алгоритмов на основе методов ньютоновского типа и $\alpha$-процессов для решения нелинейных операторных уравнений и исследование их сходимости; реализация алгоритмов в виде комплекса программ на многоядерных и графических процессорах (видеокартах) для вычислений на сетках большого размера.
	\end{block}
\end{frame}

\begin{frame}{Глава 1. Решение уравнений с монотонным оператором}
	В первой главе рассматриваются методы решения некорректных задач с нелинейным монотонным оператором. Дано обоснование двухэтапного метода на основе метода Ньютона и нелинейных аналогов $\alpha$-процессов. Приводится оценка погрешности регуляризованного решения.
\end{frame}

\begin{frame}{1.1. Регуляризованный метод Ньютона}
		Рассматривается уравнение $I$ рода с неизвестной функцией $u$, монотонным $A: H \to H$, $A^{-1}$, $A'(u)^{-1}$ разрывны в окрестности решения
		$$A(u)=f. \eqno (1.1)$$
		
		Регуляризация по схеме Лаврентьева
		$$A(u)+\alpha(u-u^0)-f_\delta=0, \eqno (1.2)$$
		где $\alpha >0$ --- параметр регуляризации, $\|f-f_\delta\|\leqslant\delta$, $u^0$ --- начальное приближение к $u_\alpha$;
		
		Итерационный процесс:
		$$ u^{k+1}=u^k-\gamma(A'(u^k)+\bar\alpha I)^{-1}(A(u^k)+\alpha(u^k-u^0)-f_\delta)\equiv{T(u^k)}, \eqno (1.3)$$
		где $A'(u^k)$ --- производная по Фреше оператора $A$ уравнения (1.1), \\ $\gamma$ --- демпфирующий множитель, $\bar{\alpha} \geqslant \alpha >0 $ --- параметры регуляризации, \\$T$ --- оператор шага.
		\scriptsize
		\let\thefootnote\relax\let\thefootnote\relax\footnotetext{\footnotesize А. Б. Бакушинский. Регуляризующий алгоритм на основе метода Ньютона -- Канторовича ... // ЖВМиМФ, 16:6 (1976). С. 1397--1404.}
%		\let\thefootnote\relax\let\thefootnote\relax\footnotetext{\footnotesize ** В. В. Васин, Е. Н. Акимова, А. Ф. Миниахметова. Итерационные алгоритмы ньютоновского типа и их приложения к обратной задаче гравиметрии // Вестник ЮУрГУ, 6:3 (2013). С. 26--37.}
\end{frame}

\begin{frame}{}
	\begin{block}{\bf Определение} Усиленное свойство Фейера для оператора $T$ означает, что для некоторого $\nu>0$ выполнено соотношение
		$$\|T(u)-z\|^2\leqslant\|u-z\|^2-\nu\|u-T(u)\|^2,\eqno (1.4)$$
		где $\forall z\in Fix(T)$ --- множеству неподвижных точек оператора $T$. Это влечет для итерационных точек $u^k$, порождаемых процессом $u^{k+1}=T(u^k)$, выполнение неравенства
		$${\|u^{k+1}-z\|}^2\leqslant{\|u^k-z\|}^2-\nu{\|u^k-u^{k+1}\|}^2.\eqno (1.5)$$
	\end{block}
	Применимость фейеровских операторов:
	\begin{itemize}
		\item построение гибридных методов;
		\item учет априорных ограничений на решение в виде систем неравенств;
	\end{itemize}
	\scriptsize
	\let\thefootnote\relax\let\thefootnote\relax\footnotetext{\footnotesize Vasin V.V., Eremin I.I. Operators and Iterative Processes of Fejer Type. Theory and Applications. Berlin/New York: Walter de Gruyter, 2009.}
\end{frame}

%\begin{frame}{\small 1.1. Оценка скорости сходимости РМН с монотонным оператором}
%	\begin{block}{\bf Теорема ~1.1.} 
%		Пусть $A$ --- монотонный оператор, для которого выполнены условия 
%		$$\forall u, v \in S(u^0;R)\quad\|A(u)-A(v)\|\leqslant N_1\|u-v\|,$$ 
%		$$\forall u, v \in S(u^0;R)\quad\|A'(u)-A'(v)\|\leqslant N_2\|u-v\|,$$
%		известна оценка для нормы производной в точке $u^0$, т.е.
%		$$	\|A'(u^0)\| \leqslant N_1,$$ 
%		$$0<\alpha \leqslant \bar\alpha,\quad\|u^0-u_\alpha\| \leqslant r,\quad r\leqslant \alpha/N_2.$$ 
%		
%		\smallskip
%		Тогда для метода Ньютона c $\gamma=1$ имеет место линейная скорость сходимости метода при аппроксимации единственного решения $u_\alpha$ регуляризованного уравнения (1.2)
%		$$\| u^{k}-u_\alpha \| \leqslant q^kr, \quad q=(1-\frac{\alpha}{2\bar\alpha}).$$
%	\end{block}
%	\let\thefootnote\relax\let\thefootnote\relax\footnotetext{\footnotesize В. В. Васин, А. Ф. Скурыдина. Двухэтапный метод построения регуляризующих алгоритмов для нелинейных некорректных задач // Труды ИММ УрО РАН Т.23 В.1 (2017), С. 57–74.}
%\end{frame}

\begin{frame}{\small 1.1. Оценка скорости сходимости РМН с монотонным оператором}
	Пусть для монотонного оператора $A$ выполнены условия 
	$$\forall u, v \in S(u^0;R) \quad \|A(u)-A(v)\|\leqslant N_1\|u-v\|,
	\|A'(u)-A'(v)\|\leqslant N_2\|u-v\|, \eqno (1.6)$$
	$$\|A'(u^0)\| \leqslant N_1, \eqno (1.7)$$
	\vskip -2 mm
	$A'(u^0)$ --- неотрицательно определенный самосопряженный оператор, $$  
	0<\alpha\leqslant\bar\alpha,\quad\bar\alpha\geqslant 4N_1,\quad \|u_\alpha-u^0\|\leqslant r,\quad r\leqslant\alpha/8N_2. \eqno (1.8)$$
	\vskip -2 mm
	\begin{block}{\bf Теорема ~1.3.}
		Пусть выполнены условия (1.6)--(1.8). Тогда при
		$\gamma<\frac{\alpha\bar\alpha}{2(N_1+\alpha)^2}$
		оператор шага $T$ метода Ньютона при
		$$\nu=\frac{\alpha\bar\alpha}{2\gamma(N_1+\alpha)^2}-1$$
		удовлетворяет неравенству сильной фейеровости, последовательность итераций $u^k$ является сильно фейеровской и имеет место сходимость
		$$\lim_{k\to\infty}\|u^k-u_\alpha\|=0.$$
		Если параметр $\gamma$ принимает значение ${\gamma}_{opt}=\frac{\alpha\bar\alpha}{4(N_1+\alpha)^2},$ то справедлива оценка $$\|u^k-u_\alpha\|\leqslant q^k r, \quad q=\sqrt{1-\frac{{\alpha}^2}  {16(N_1+\alpha)^2}}.$$
	\end{block}
\end{frame}

\begin{frame}{1.2. Нелинейные аналоги альфа-процессов}
	Впервые $\alpha$-процессы для линейного  самосопряженного положительно определенного оператора  были предложены М.~А.~Красносельским и др. (1969). \\
	Для нелинейного оператора итерационный процесс запишем в виде:
	$$u^{k+1}=u^k-\beta_k(A(u^k)-f_{\delta}).$$ \\
	Используя разложение Тейлора исходного уравнения в точке $u^k$, получим
	$$A'(u^k)u^k=F^k, \quad F^k=f_\delta+A'(u^k)u^k-A(u^k).$$ Параметр $\beta_k$ находим из условия минимума соответствующих функционалов.
%	\begin{itemize}
%		\item регуляризованный метод минимальной ошибки (ММО): $$\min_{\beta}{\|u^k-\beta(A(u^k)-f_{\delta})-z\|^2},\quad u^{k+1} =u^k - \frac{\langle B^{-1}(u^k)S_\alpha(u^k), S_\alpha (u^k)\rangle}{\|S_\alpha(u^k)\|^2}S_\alpha(u^k),$$
%		\item регуляризованный метод минимальной ошибки (МНС):
%		$$\min_{\beta}\{<A'(u^k)u^{k+1},u^{k+1}>-2<u^{k+1},F(u^k)>\}, u^{k+1} =u^k - \frac{\langle S_\alpha(u^k), S_\alpha (u^k)\rangle}{\langle B(u^k)S_\alpha(u^k), S_\alpha(u^k)\rangle}(A(u^k)+\alpha(u^k-u^0)-f_\delta)$$
%		\item регуляризованный метод минимальных невязок (ММН):
%		$$\min_{\beta}\{\|A'(u^k)(u^k-\beta(A(u^k)-f_{\delta})-F(u^k)\|^2\},\quad u^{k+1} =u^k - \frac{\langle B(u^k)S_\alpha(u^k), S_\alpha (u^k)\rangle}{\|B(u^k)S_\alpha(u^k)\|^2}(A(u^k)+\alpha(u^k-u^0)-f_\delta),$$
%		где $B(u^k)=A'(u^k)+\bar{\alpha}I, \quad S_\alpha(u^k)=A(u^k)+\alpha(u^k-u^0)-f_\delta$.
%	\end{itemize}
	\let\thefootnote\relax\let\thefootnote\relax\footnotetext{\footnotesize М. А. Красносельский, Г. М. Забрейко, П. П. Забрейко и др. Приближенное решение операторных уравнений // М.: Наука, 1969.}
\end{frame}

\begin{frame}{Модифицированные варианты на основе аналогов $\alpha$-процессов}
	Производная оператора задачи вычисляется 
	в начальном приближении:
	%$$ u^{k+1}=u^k-\gamma\frac{<(A'(u^0)+\bar\alpha I)^{\varkappa}S_\alpha(u^k), S_\alpha(u^k)>}{<(A'(u^0)+\bar\alpha I)^{\varkappa+1}S_\alpha(u^k), S_\alpha(u^k)>}S_\alpha(u^k)\equiv{T(u^k)},\eqno (3.3)$$
	%$\varkappa=-1,0,1$.
	\begin{itemize}
		\item ММО: $$u^{k+1}=u^k-\gamma\frac{\big\langle(A'(u^0)+\bar{\alpha}I)^{-1}S_\alpha(u^k),S_\alpha(u^k)\big\rangle}{\|S_\alpha(u^k)\|^2}S_\alpha(u^k),$$
		\item МНС:
		$$u^{k+1}=u^k-\gamma\frac{\|S_\alpha(u^k)\|^2}{\big\langle(A'(u^0))S_\alpha(u^k), S_\alpha(u^k)\big\rangle}S_\alpha(u^k),$$
		\item ММН:
		$$u^{k+1}=u^k-\gamma\frac{\big\langle(A'(u^0))S_\alpha(u^k), S_\alpha(u^k)\big\rangle}{\|(A'(u^0))S_\alpha(u^k)\|^2}S_\alpha(u^k).$$
	\end{itemize}
	
	\smallskip
	Данный прием является более экономичным с точки зрения численной реализации, так как позволяет снизить вычислительные затраты по времени.
	\let\thefootnote\relax\let\thefootnote\relax\footnotetext{\footnotesize Vasin V. V. Regularized modified alpha-processes for nonlinear equations with monotone operators // Dokl. Math. -- 2016. -- Vol.469, pp. 13--16}
\end{frame}

\begin{frame}
	\begin{itemize}
		\item регуляризованный метод минимальной ошибки (ММО): $$\min_{\beta}{\|u^k-\beta(A(u^k)-f_{\delta})-z\|^2},\quad u^{k+1} =u^k - \gamma\frac{\langle B^{-1}(u^k)S_\alpha(u^k), S_\alpha (u^k)\rangle}{\|S_\alpha(u^k)\|^2}S_\alpha(u^k),$$
		\item регуляризованный метод наискорейшего спуска (МНС):
		$$\min_{\beta}\{\langle A'(u^k)u^{k+1},u^{k+1}\rangle -2\langle u^{k+1},F(u^k)\rangle \},$$
		$$ u^{k+1} =u^k - \gamma\frac{\langle S_\alpha(u^k), S_\alpha (u^k)\rangle}{\langle B(u^k)S_\alpha(u^k), S_\alpha(u^k)\rangle}S_\alpha(u^k)$$
		\item регуляризованный метод минимальных невязок (ММН):
		$$\min_{\beta}\{\|A'(u^k)(u^k-\beta(A(u^k)-f_{\delta})-F(u^k)\|^2\},$$$$ u^{k+1} =u^k - \gamma\frac{\langle B(u^k)S_\alpha(u^k), S_\alpha (u^k)\rangle}{\|B(u^k)S_\alpha(u^k)\|^2}S_\alpha(u^k),$$
		\vskip 3mm
		где $B(u^k)=A'(u^k)+\bar{\alpha}I, \quad S_\alpha(u^k)=A(u^k)+\alpha(u^k-u^0)-f_\delta$, 
		\vskip 2mm
		$\gamma$ --- демпфирующий множитель.
		$$ u^{k+1}=u^k-\gamma\frac{\langle(A'(u^k)+\bar\alpha I)^{\varkappa}S_\alpha(u^k), S_\alpha(u^k)\rangle}{\langle(A'(u^k)+\bar\alpha I)^{\varkappa+1}S_\alpha(u^k), S_\alpha(u^k)\rangle}S_\alpha(u^k)\equiv{T(u^k)},$$
		$\varkappa=-1,0,1$.
	\end{itemize}
\end{frame}

\begin{frame}{\small 1.2. Оценка скорости сходимости нелинейных аналогов $\alpha$-процессов}
		Пусть для монотонного оператора $A$ выполнены условия \\ $\forall u, v \in S(u^0;R)$ $\|A(u)-A(v)\|\leqslant N_1\|u-v\|$, $\|A'(u)-A'(v)\|\leqslant N_2\|u-v\|$,	\\ $\|A'(u^0)\| \leqslant N_1$, $\alpha \leqslant \bar\alpha$, $\bar\alpha \geqslant N_1$, \\для ММО $A'(u^0)$ --- самосопряженный оператор, $\|u^0-u_\alpha\| \leqslant r$, $r\leqslant \alpha/8N_2.$
	\begin{block}{\bf Теорема ~1.5.}
		Пусть выполнены условия, $\mu _\varkappa$ вычисляется для каждого из трех $\alpha$-процессов. Тогда при
		$$\gamma <\frac{2}{\mu _\varkappa}\quad (\varkappa=-1,0,1)$$
		для последовательности $\{u^k\}$, порождаемой $\alpha$-процессом при соответствующем $\varkappa$, имеет место сходимость $\lim_{k\to\infty}\|u^k-u_\alpha\|=0, $ а при 
		$\gamma^{opt}=\frac{1}{\mu_\varkappa}$
		справедлива оценка $\|u^k-u_\alpha\|\leqslant q{_\varkappa^k}\cdot r,$ \ \ где
		$$
		q_{-1}=\sqrt{1-\frac{\alpha^2}{16(N_1+\alpha)^2}}, \quad q_0=\sqrt{1-\frac{\alpha^2\bar\alpha^2}{(N_1+\alpha)^2(N_1+\bar\alpha)^2}}, \quad $$$$q_1=\sqrt{1-\frac{\alpha^2\bar\alpha^4}{(N_1+\bar\alpha)^4}}.
		$$
	\end{block}
\end{frame}

\begin{frame}{1.3. Оценка погрешности двухэтапного метода для монотонного оператора}
%	При условии монотонности оператора и истокообразной представимости решения $\hat{u}$ исходного уравнения
%	$$u^0-\hat{u}=A'(\hat{u})v,$$
%	справедлива оценка погрешности регуляризованного решения
%	$$\|u_\alpha^{\delta}-\hat{u}\|\le\|u_\alpha^{\delta}-u_\alpha\|+\|u_\alpha-\hat{u}\|\le\frac{\delta}{\alpha}+k_0\alpha,$$
%	где $k_0=(1+N_2\|v\|/2)\|v\|$, $u_\alpha^{\delta}$, $u_\alpha$ -- решения уравнения $\eqref{equ2}$ с возмущенной $f_\delta$ и точной $f$ правой частью уравнения $\eqref{equ1}$ соответственно. Минимизируя правую часть соотношения $\eqref{est5.2}$ по $\alpha$, имеем $\alpha(\delta)=\sqrt{\delta /k_0}$, что дает оценку
%	$$\|u_{\alpha(\delta)}^{\delta, k}-\hat{u}\|\le 2\sqrt{\delta k_0}
%	$$
%	Для методов главы 1 имеются оценки 
%	$\|u_{\alpha(\delta)}^{\delta, k}-u_{\alpha(\delta)}^{\delta}\|\le q^k(\delta)r$
$$\|u_{\alpha(\delta)}^{\delta, k}-\hat{u}\|\le\|u_{\alpha(\delta)}^{\delta, k}-u_{\alpha(\delta)}^{\delta}\|+\|u_{\alpha(\delta)}^{\delta}-\hat{u}\|\le rq^k(\delta)+ 2\sqrt{k_0\delta}.$$
	Получена оценка погрешности двухэтапного метода с монотонным оператором:
	\begin{equation*}
	\|u_{\alpha(\delta)}^{\delta, k}-\hat{u}\|\leqslant 4\sqrt{k_0 \delta}
	\end{equation*}
	при выборе числа итераций по правилу $$k(\delta)=\left[\frac{\ln(2\sqrt{k_0\delta}/r)}{\ln q(\delta)}\right],$$
	где приближение $u_{\alpha(\delta)}^{\delta, k}$ зависит от $\delta$, $\hat{u}$ --- решение исходного уравнения, $k_0=(1+N_2\|v\|/2)\|v\|$ \\
	$u^0-\hat{u}=A'(\hat{u})v$ --- истокообразная представимость решения.
	{ 
	\let\thefootnote\relax\let\thefootnote\relax\footnotetext{\footnotesize U. Tautenhahn. On the method of Lavrentiev regurarization for nonlinear ill-posed problems // Inverse Problem. 2002. Vol. 91, №1. P. 191–207.}
	\let\thefootnote\relax\let\thefootnote\relax\footnotetext{\footnotesize В. К. Иванов, В. В. Васин, В. П. Танана. Теория линейных некорректных задач и её приложения. М.: Наука, 1978. --- 206 с.}
	}
\end{frame}

\begin{frame}{1.3. Иллюстрирующий пример}
%	Рассматривается ДУ с $x(t)$, $y(t)$, $t\in[0, 1]$ с заданной константой $c_0>0$
%	$$	\frac{dy}{dt}=x(t)y(t), \quad y(0)=c_0,$$
%	где $x(t), y(t)\in L^2[0,1]$. Интегрируя ДУ, приходим к нелинейному операторному уравнению
Рассматривается нелинейное операторное уравнение
	$$	F(u)=y, \quad [F(u)](t)=c_0 e^{\int_{0}^{t}u(\tau)d\tau},$$
%	где $$[F(u)](t)=c_0 e^{\int_{0}^{t}u(\tau)d\tau}$$
	$F$ действует из $L^2[0,1]$ в $L^2[0,1]$. 

	
	Правая часть задана с шумом $$y^\delta(t)=y(t)e^{\frac{\delta}{5} sin(t/{\delta}^2)},$$
	При $y^\delta\to y$ в $L^2[0,1]$,  $\|u-u^\delta\|=\|\frac{1}{\delta}cos(t/{\delta}^2)\|\to\infty$ при $\delta\to 0$.
	\vskip 0.3cm

	{\textbf{\color{blue}Данные:}} $\delta=0.1$, $\|y-y^{\delta}\|=0.07<\delta$. Точное решение --- функция $z(t)=t^2$. 
	\vskip 1mm
	Начальное приближение $u^0(t)=0$, $\gamma=1$, $\bar\alpha=1$, $\alpha=10^{-3}$, критерий останова
	\vskip 1mm
	 $\frac{\|u^k-z\|}{\|z\|}\leqslant\varepsilon=0.25$, где $u^k$ --- приближение на $k$-й итерации.
	\vskip 0.3cm
	{\textbf{\color{blue}Результаты:}} проверена применимость методов, получено приближенное решение.	

	\let\thefootnote\relax\let\thefootnote\relax\footnotetext{\footnotesize U. Tautenhahn. On the method of Lavrentiev regularization for nonlinear 	ill-posed problems // Inverse Problems. 2002. pp. 191--207.}
\end{frame}

\begin{frame}{Глава 2. Решение уравнений с немонотонным оператором}
	
	Во второй главе обоснована сходимость итераций РМН, ММО, МНС, ММН к регуляризованному решению в~конечномерном случае без требования монотонности оператора $A$ исходного уравнения.
	
	\smallskip
	Приведены результаты расчетов.
\end{frame}
 
\begin{frame}{\small 2.1. Скорость сходимости РМН с немонотонным оператором}
$A\colon \mathbb{R}^n\to \mathbb{R}^n$, $\forall i,j$ $\lambda _i>0$, $\lambda _i\neq\lambda_j$, $i\neq j$, $A'(u)+\bar\alpha I =S(u)\Lambda(u) S^{-1}(u)$, $\mu(S(u))=\|S(u)\|\cdot\|S^{-1}(u)\|.$
Имеется оценка:
	$$\|(A'(u)+\bar\alpha I)^{-1}\|\leqslant \frac{\mu (S(u))}{\bar\alpha+\lambda_{min}} \leqslant \frac{\mu(S(u))}{\bar\alpha}.$$
	\vskip -1 mm
	\begin{block}{\bf Теорема ~2.1.}
		Пусть выполнены условия, а также: \quad $$\sup\{\mu(S(u)): u\in S(u_\alpha;r)\}\leqslant\bar S <\infty,$$ 
		$$\|A(u)-A(v)\|\leqslant N_1\|u-v\|,\quad
		\|A'(u)-A'(v)\|\leqslant N_2\|u-v\|, \quad \forall u, v \in S(u^0;R),$$
		$$\|A'(u^0)\| \leqslant N_1.$$
		$A'(u^0)$ --- симметричная матрица, $0<\alpha\leqslant\bar\alpha$, $\bar\alpha\geqslant 4N_1$, $\|u^0-u_\alpha\| \leqslant r$, $r\leqslant\alpha/8N_2\bar S$. Тогда для метода (1.3) справедливо заключение теоремы (1.3), где
		$$\gamma<\frac{\alpha\bar\alpha}{2(N_1+\alpha)^2\bar S^2},
		\quad
		{\gamma}^{opt}=\frac{\alpha\bar\alpha}{4(N_1+\alpha)^2\bar S^2},$$ 
		$$\|u^k-u_\alpha\|\leqslant q^k r, \quad q=\sqrt{1-\frac{\alpha ^2}{16(N_1+\alpha)^2\bar S^2}}.$$
	\end{block}
\end{frame}

\begin{frame}{\small 2.2. Скорость сходимости аналогов $\alpha$-процессов с немонотонным оператором}
	\begin{block}{\bf Теорема ~2.3.}
		Пусть выполнены условия (пред. слайд), $0<\alpha\leqslant\bar\alpha$, $\|u^0-u_\alpha\| \leqslant r$, для MMO: $r\leqslant\alpha /6\bar SN_2$, $\bar\alpha \geqslant N_1$, \ MHC: $0<\alpha\leqslant\bar\alpha$, $\quad r\leqslant\alpha /3N_2$,
		MMH: $0<\alpha\leqslant\bar\alpha, \quad r\leqslant\alpha /6N_2$. 
		
		Тогда при $\gamma<2/\mu _\varkappa$, $\varkappa=-1,0,1$, с соответствующими $\mu _\varkappa$, последовательности ${u^k}$, порождаемые процессом (1.6) при $\varkappa=-1,0,1$, сходятся к $u_\alpha$, т.е., $$\lim_{k\to\infty}\|u^k-u_\alpha\|=0,$$ а при $
		\gamma^{opt}=\frac{1}{\mu_\varkappa}$
		справедлива оценка $$\|u^{k+1}-u_\alpha\|\leqslant q{_\varkappa^k}r,$$ где
		$$q_{-1}=\sqrt{1-\frac{\alpha^2}{64\bar S^2(N_1+\alpha)^2}}, \quad q_0=\sqrt{1-\frac{\alpha^2\bar\alpha^2}{36(N_1+\alpha)^2(N_1+\bar\alpha)^2}},$$
		$$q_1=\sqrt{1-\frac{\alpha^2\bar\alpha^6}{36(N_1+\alpha)^2(N_1+\bar\alpha)^6}}.$$
	\end{block}
\end{frame}


\begin{frame}{\small 2.3. Модифицированные аналоги $\alpha$-процессов с немонотонным оператором}
	Пусть выполнены условия:
	$\forall u, v \in S(u^0;R)$ $\|A(u)-A(v)\|\leqslant N_1\|u-v\|$, $\|A'(u)-A'(v)\|\leqslant N_2\|u-v\|,$
	$\|A'(u^0)\| \leqslant N_1,$
	$A'(u^0)$ --- симметричная матрица, спектр которого состоит из неотрицательных различных собственных значений,
	$0<\alpha\leqslant\bar{\alpha},\quad \bar{\alpha}\geqslant N_1$, $\|u^0-u_\alpha\| \leqslant r$, $r\leqslant\alpha/6N_2$.
	\vskip -1 mm	
	\begin{block}{Теорема 2.5}
		Пусть выполнены условия. Тогда при
		$$\gamma <\frac{2}{\mu _\varkappa}\quad (\varkappa=-1,0,1)$$
		для последовательности $\{u^k\}$, порождаемой модифицированным $\alpha$-процессом при соответствующем $\varkappa$, имеет место сходимость $\lim_{k\to\infty}\|u^k-u_\alpha\|=0, $ а при 
		$\gamma^{opt}=\frac{1}{\mu_\varkappa}$
		справедлива оценка $$\|u^k-u_\alpha\|\leqslant q^k r,$$ где
		$$q=\sqrt{1-\frac{9\alpha^2}{64(N_1+\alpha)^2}}$$
	\end{block}
\end{frame}

\begin{frame}
	\begin{block}{\bf Замечание~2.4} Предложенный подход к получению оценок скорости сходимости итерационных процессов полностью переносится на случай, когда спектр матрицы $A'(u^k)$, состоящий из различных вещественных значений, содержит набор малых по абсолютной величине отрицательных собственных значений. Пусть $\lambda _1$ --- наименьшее отрицательное собственное значение с  модулем $|\lambda_1|$ и $\bar\alpha -|\lambda _1|=\bar\alpha _1<\alpha^*$. Тогда оценка 
		$$\|(A'(u)+\bar\alpha I)^{-1}\|\leqslant \frac{\mu (S(u))}{\bar\alpha+\lambda_{min}} \leqslant \frac{\mu(S(u))}{\bar\alpha},$$ 
		трансформируется в неравенство
		$$\|(A'(u^k)+\bar\alpha I)^{-1}\|\leqslant\frac{\mu(S(u^k))}{\bar\alpha^*}\leqslant\frac{\bar S}{\bar\alpha^*}$$
		Все теоремы остаются справедливыми при замене $\bar\alpha$ на $\bar\alpha^*$.
	\end{block}
\end{frame}

\begin{frame}{2.3. Оценка невязки двухэтапного метода с немонотонным оператором}
	В конечномерном случае для оператора $A'(u)$ с положительным спектром можно установить оценку для невязки --- основной характеристики точности метода при решении задачи с реальными данными.
	$$\|A(u_{\alpha(\delta)}^{\delta, k})-f_\delta\|\le\|A(u_{\alpha(\delta)}^{\delta, k})-A(u_{\alpha(\delta)}^{\delta})\|+\|A(u_{\alpha(\delta)}^{\delta})-f(\delta)\|\le N_1 r q^k(\delta)+\alpha(\delta)m,$$
	где $u_{\alpha(\delta)}^{\delta}$ --- регуляризованное решение, для $\alpha(\delta)$ ограничена величина $\|u_{\alpha(\delta)}^{\delta}-u^0\|\leqslant m <\infty$.
	
	Приравнивая слагаемые в правой части, получаем правило выбора числа итерации
	$k(\delta)=\left [\ln(m\delta^p/N)/\ln q(\delta)\right ],$ при котором справедлива оценка для невязки двухэтапного метода
	$$\|A(u_{\alpha(\delta),k}^{\delta})-f_\delta\|\leqslant 2m\delta^p, \quad \alpha(\delta)=\delta^p.$$ 
\end{frame}

\begin{frame}{2.4. Решение модельной задачи магнитометрии}
	\vskip -4 mm
	\begin{equation*}\begin{aligned}
	\left[A(u)\right](x,y)=\frac{\mu_0}{4\pi}\Delta J  \bigg\{&\iint_{D} \frac{H}{[(x-x')^2+(y-y')^2+H^2]^{3/2}}dx'dy' \notag\\
	- &\iint_{D} \frac{u(x',y')}{[(x-x')^2+(y-y')^2+u^2(x',y')]^{3/2}}dx'dy' \bigg\}= B_z(x,y,0),
	\end{aligned} \end{equation*}
	где $\mu_0/{4\pi}=10^{-7}$ Гн/м --- магнитная постоянная, $\Delta J$ --- скачок $z$-компоненты вектора намагниченности, $z=H$ --- асимптотическая плоскость, $ B_z(x,y,0)$ --- функция аномального поля, $z=u(x,y)$ --- искомая функция%, описывающая поверхность раздела сред с различными свойствами намагниченности.
	
	\smallskip
	Точное решение уравнения магнитометрии задается формулой%$^*$%\textbf{\color{red}(Акимова, Мисилов, 2014-2016)}
	$$\hat{u}(x,y)=5-2e^{-(x/10-3.5)^6-(y/10-2.5)^6}-3^{-(x/10-5.5)^6-(y/10-4.5)^6} \quad(\textup{км}),$$
	\vskip -2 mm
	\textbf{\color{blue}Результаты.} Число обусловленности $\mu(A'_n(u_n^k))\approx 1.8\cdot 10^7$, спектр неотрицательный, состоящий из различных собственных значений, $\bar\alpha=10^{-2}$, $\alpha = 10^{-4}$, $\gamma=1$, относительная погрешность $\varepsilon < 10^{-2}$ получена за 4--5 итераций, у модифицированных меньше время счета.
\let\thefootnote\relax\let\thefootnote\relax\footnotetext{\footnotesize Е. Н. Акимова, В. Е. Мисилов, А. Ф. Скурыдина. Параллельные алгоритмы решения структурной обратной задачи магнитометрии ... // Вестник УГАТУ,  Т.18, №2 (2014). С. 208--217.}
\end{frame}

%\begin{frame}{}
%	\textbf{\color{blue}Цель:} проверить заключения теорем главы 2 на примере решения обратной задачи магнитометрии.
%	Число обусловленности $\mu(A'_n(u_n^k))\approx 1.8\cdot 10^7$, спектр неотрицательный, состоящий из различных собственных значений, $\bar\alpha=10^{-2}$, $\alpha = 10^{-4}$, $\gamma=1$, $\varepsilon < 10^{-2}$
%	
%	\begin{table}
%		\centering
%		{\scriptsize Табл.2. Относительные нормы невязок, числа итераций и времена счета в задаче магнитометрии}
%		
%		\smallskip
%\begin{tabular}{|c|c|c|c|c|}
%\hline
%Методы                    & ММО    & МНС    & ММН    & РМН    \\ \hline
%\multirow{2}{*}{$\Delta$} & 0.0636 & 0.0699 & 0.0802 & 0.0368 \\ \cline{2-5} 
%                          & 0.0569 & 0.0575 & 0.0595 & 0.0369 \\ \hline
%\multirow{2}{*}{N}        & 4      & 4      & 4      & 5      \\ \cline{2-5} 
%                          & 4      & 4      & 4      & 5      \\ \hline
%\multirow{2}{*}{T (сек)}  & 10     & 6      & 6      & 22     \\ \cline{2-5} 
%                          & 5      & 3      & 3      & 3      \\ \hline
%\end{tabular}
%	\end{table}
%	\textbf{\color{blue}Вывод:} число итераций для модифицированных методов больше, чем для немодифицированных, но затраты машинного времени меньше.
%\end{frame}

\begin{frame}{Глава 3. Покомпонентные методы ньютоновского типа решения обратных структурных задач гравиметрии и магнитометрии}
	В третьей главе предложены покомпонентные методы типа Ньютона и типа Левенберга -- Марквардта, а также вычислительная оптимизация метода Ньютона. 
	
	Параллельные алгоритмы реализованы в виде комплекса программ на многоядерных и графических процессорах (видеокартах) для вычислений на сетках большого размера. Приводятся результаты расчетов модельных задач на многоядерных процессорах и графических ускорителях. 
	
\end{frame}

\begin{frame}{Постановка обратной задачи гравиметрии восстановления поверхности раздела сред}
	Рассмотрим уравнение гравиметрии в декартовой системе координат с осью $z$, направленной вниз 
\vskip -2 mm
	\begin{equation*}
	\begin{aligned}
	A(u)=f\Delta\sigma \bigg\{ &\iint_{D} \frac{1}{[(x-x')^2+(y-y')^2+H^2]^{1/2}}dx'dy' \\
	- &\iint_{D} \frac{1}{[(x-x')^2+(y-y')^2+u^2(x',y')]^{1/2}}dx'dy'\bigg\}=\Delta g(x,y,0),
	\end{aligned} 
	\end{equation*}
\vskip -3 mm
	\begin{columns}
		\column{0.5\linewidth}
		где $f$ --- гравитационная постоянная, равная $6.67\cdot10^{-8}$ см$^3/$г$\cdot c^2$, $\Delta\sigma=\sigma_2-\sigma_1$ --- скачок плотности на поверхности раздела сред $u(x,y)$, $\Delta g(x,y,0)$ --- аномальное гравитационное поле,
		$D=\{c\leqslant x \leqslant d, a\leqslant y \leqslant b\}$
		\column{0.5\linewidth}
		\begin{figure}[h]
			\centering
			\includegraphics[height=4cm]{grav_illust.png}
			\label{fig:twolayer}
		\end{figure}
	\end{columns}
\end{frame}

\begin{frame}{Вычислительная оптимизация метода Ньютона}
	Прозводная оператора $A$ в точке $u^0(x,y)$ определяется формулой:
	\begin{itemize}
		\item в задаче гравиметрии
		$$ [A'(u^0)]h=\iint_{D} \frac{u^0(x',y')h(x',y')}{[(x-x')^2+(y-y')^2+(u^0(x',y'))^2]^{3/2}}dx'dy',$$
		\item в задаче магнитометрии
		$$ [A'(u^0)]h=\iint_{D} \frac{(x-x')^2+(y-y')^2-2(u^0(x',y'))^2}{[(x-x')^2+(y-y')^2+(u^0(x',y'))^2]^{5/2}}h(x',y')dx'dy'.$$
	\end{itemize}
	
	Матрица $A'_n(u^0)$ имеет диагональное преобладание.
	При расчетах используется лента матрицы $A'_n(u^k)$.
	\begin{figure}
		\includegraphics[width=0.32\linewidth, height=0.32\textheight]{Matrix}
	\end{figure}
	\centering \scriptsize Схема матрицы производной оператора $A$
\end{frame}

\begin{frame}
	\begin{block}{Замечание 3.2.}
		В структурных обратных задачах грави- магнитометрии при решении методом Ньютона без существенной потери точности можно не учитывать значения элементов, отстоящих от диагонали далее, чем на  $\beta$-ю часть  размерности матрицы производной, то есть те значения $a_{ij}$, для которых  $j \in \{i-h(\beta),..i+h(\beta)\} $, где $h(\beta)$ --- полуширина ленты матрицы, $i, j$ --- индекс элемента.
		
		В формуле (1.3) вместо матрицы $A'(u^k)=\{a_{i,j}\}$ используется диагональная матрица $\overline{A}=\{\overline{a_{i,j}}\}, \quad \overline{a_{i,j}}=
		\begin{cases}
		a_{i,j},\quad  |i-j|\leq\beta \cdot n, \\
		0, \quad \quad | i-j|>\beta \cdot n. 
		\end{cases}$
	\end{block}
	\let\thefootnote\relax\let\thefootnote\relax\footnotetext{\footnotesize Akimova E., Miniakhmetova A., Martyshko M. Optimization and parallelization of Newton type methods for solving structurial gravimetry and magnetometry inverse problems // EAGE Geoinformatics 2014}
\end{frame}

\begin{frame}{3.1. Покомпонентный метод типа Ньютона}
	Рассмотрим нелинейное интегральное уравнение:
	$$[A(u)](t)=\int_{a}^{b}K(t,s,u(s))ds=f(t).$$
Поправка в методе Ньютона $\Delta u^k$ определяется из решения линейного интегрального уравнения:
$$A'(u^k)\Delta u^k=\int_{a}^{b}K'_u(t,s,u^k(s))\Delta u^k(s)ds=A(u^k)-f.$$
После дискретизации получаем систему уравнений:
$$\sum\limits_{j=1}^{n}h K'_u(t,s_j,u^k(s_j))\Delta u^k(s_j)=[A(u^k)](t)-f(t)|_{t=t_i}, \quad 1\leqslant i\leqslant n.$$
Теперь в каждом $i$-м уравнении полагаем, что все поправки $\Delta u^k(s_j)=\Delta u^k(s_i)$, то есть все компоненты вектора $\{\Delta u^k(s_j)\}_1^n$ одинаковы. $\Delta u^k(s_i)$ вынесем за знак суммы.
$$\Delta u^k(s_i)=\frac{[A(u^k)](t)-f(t)|_{t=t_i}}{\sum\limits_{j=1}^{n}h K'_u(t,s_j,u^k(s_j))}, \quad 1\leqslant i\leqslant n.$$
\end{frame}

%\begin{frame}{3.1. Покомпонентный метод типа Ньютона}
%	В уравнении гравиметрии
%	$$A(u)=f,$$
%	обозначим $A(u)=\int_{a}^{b}\int_{c}^{d}K(x,y, x',y',u^k(x,y))dxdy$ --- интегральный оператор.
%	
%	Итерации в методе Ньютона строятся по схеме
%	$$A'(u^k)(\Delta u^k)=-(A(u^k)-f),$$ где $\Delta u^k=u^{k+1}-u^k$.
%	Для задачи гравиметрии
%	$$f\Delta\sigma\int_{a}^{b}\int_{c}^{d}K'_u(x,y, x',y',u^k(x,y))\Delta u^k(x,y) dxdy=A(u(x',y'))-f(x',y').$$
% %\textbf{Замечание 3.1. (ИГФ)} {\color{blue}  На значение гравитационного поля в точке $(x',y')$ наибольшее влияние оказывает глубина залегания поверхности в точке $(x',y')$ .} 
%
%\textbf{\color{blue}Заменим} $\Delta u^k(x,y)=\Delta u^k(x',y')=const$ относительно переменных интегрирования, т.к. изменение в правой части в основном зависит от значений $\Delta u^k(x',y')$, и вынесем за знак интеграла $\Delta u^k(x',y')$
%	$$f\Delta\sigma{\color{blue}(\Delta u^k(x',y'))}\int_{a}^{b}\int_{c}^{d}K'_u(x,y, x',y',u^k(x,y)) dxdy\approx A(u(x',y'))-f(x',y').$$
%	$$f\Delta\sigma \Delta u^k_{m,l}\sum\limits_{i=1}^{M}\sum\limits_{j=1}^{N}
%	\Delta x\Delta y K'_u(x_j,y_i,x_k,y_l,u_{m,l})\approx \{A[u^k]\}_{m,l}-f_{m,l}.$$
%\end{frame}

\begin{frame}{}
	 Регуляризованный метод Ньютона:
	$$ u^{k+1}=u^k-\gamma{\color{red}(A'(u^k)+\bar\alpha I)^{-1}}(A(u^k)+\alpha(u^k-u^0)-f_\delta).$$

	\vskip 0.1 cm
	Покомпонентный метод типа Ньютона имеет вид:
	$$u^{k+1}(x',y')=u^k(x',y')-\gamma{\color{blue}\frac{1}{\varPsi(x',y')}}\big([A(u^k)](x',y')-f(x',y')\big),$$
	где
	\vskip -5 mm $$\varPsi(x',y')=f\Delta\sigma\int_{a}^{b}\int_{c}^{d}K'_u(x,y, x',y',u^k(x,y)) dxdy.$$
	\vskip 0.3 cm

	Регуляризованный покомпонентный метод типа Ньютона имеет вид:
	$$u^{k+1}(x',y')=u^k(x',y')-\gamma{\color{blue}\frac{1}{\varPsi(x',y')+\bar{\alpha}}}([A(u^k)](x',y')+$$ 
\vskip -3 mm
	$$+\alpha (u^k(x',y')-u^0(x',y'))-f_\delta(x',y')),$$
	где $\alpha, \bar{\alpha}$ --- положительные параметры регуляризации.
	
	В дискретной записи
	$$u_{m,l}^{k+1}=u_{m,l}^k-\frac{1}{\psi_{m,l}^k+\bar\alpha}([A_n(u^k)]_{m,l} + \alpha(u^k-u^0) -f_{m,l}),\quad 1\le m \le M, \quad 1\le l \le N,$$
	где $$\psi_{m,l}^k=f\Delta\sigma\sum\limits_{i=1}^{M}\sum\limits_{j=1}^{N}
	\Delta x\Delta y\frac{u_{ij}}{[(x_k-x'_j)^2+(y_l-y'_i)^2+(u_{ij})^2]^{3/2}}.$$
	\let\thefootnote\relax\let\thefootnote\relax\footnotetext{\footnotesize Akimova E., Skurydina A. A componentwise Newton type method for solving the structural inverse gravity problem // EAGE Geoinformatics 2015}
\end{frame}

\begin{frame}{Постановка обратной задачи гравиметрии восстановления нескольких поверхностей раздела}
	\begin{columns}
\column{0.6\linewidth}
	Суммарное гравитационное поле получаем путем сложения гравитационных полей от каждой поверхности:
	\begin{equation*}
	\begin{aligned}
	& A(u)=\sum_{l=1}^{L}f\Delta\sigma_l\frac{1}{4\pi}\times \\
&\times\iint_D\bigg\{\frac{1}{[(x-x')^2+(y-y')^2+u_l^2(x,y)]^{1/2}} \\
	&-\frac{1}{[(x-x')^2+(y-y')^2+H_l^2]^{1/2}}\bigg\}=\Delta g(x',y'),
	\end{aligned}
	\end{equation*}

	где $L$~--- число границ раздела, \\ 
	$f=6.67\cdot10^{-8}$ см$^3/$г$\cdot c^2$, $\Delta\sigma_l=\sigma_l-\sigma_{l-1}$ --- скачок плотности на $l$-й поверхности раздела сред $u_l(x,y)$, $\Delta g(x,y,0)$ --- аномальное гравитационное поле.
\column{0.4\linewidth}
%\begin{figure}[h]
		\centering
		\includegraphics[height=6.0cm]{whitegrav.png}
		\label{fig:multlayer}
%	\end{figure}
\end{columns}

%	Регуляризованный метод Левенберга -- Марквардта %\textbf{\color{red}(Васин, 2012)}
%	$$	u^{k+1}=u^k-\gamma[A'(u^k)^*A'(u^k)+\alpha I]^{-1} [A'(u^k)^*(A(u^k)-f_\delta)]$$
\end{frame}

%\begin{frame}{Модель многослойной среды}
%	\begin{figure}[h]
%		\centering
%		\includegraphics[height=7.0cm]{whitegrav.png}
%		\label{fig:multlayer}
%	\end{figure}
%	\centering
%	%Рис.3. Модель многослойной среды.
%\end{frame}

\begin{frame}{3.2. Покомпонентный метод типа \\ Левенберга -- Марквардта}
	\vskip -2 mm
	Регуляризованный метод Левенберга -- Марквардта с весовыми множителями %\textbf{\color{red}(Васин, 2012)}
	$$	u^{k+1}=u^k-\gamma{\color{red}[A'(u^k)^T A'(u^k)+\alpha I]^{-1} }\Lambda[A'(u^k)^T(A(u^k)-f_\delta)]$$
	По аналогии с ПМН, получим покомпонентный метод типа Левенберга -- Марквардта:
	\vskip -3 mm
	$$ u_l^{k+1}=u_l^k-\gamma{\color{blue}\frac{1}{\varphi_l(x',y')+\bar{\alpha}}}\Lambda[ A'(u_l^k)^T(A(u^k)-f_\delta)],$$
	где $l$ --- номер границы раздела, $l=1,..,L$, $\Lambda$ --- диагональный весовой оператор, 
	\begin{equation*}
	\begin{aligned}
	\varphi_l(x',y')=\bigg[ f\Delta\sigma\int_{c}^{d}\int_{a}^{b}
	K'_u(x,y, x', y', u_l^k(x',y'))dydx\bigg] \notag \\ \times\bigg[f\Delta\sigma\int_{a}^{b}\int_{c}^{d}K'_u(x,y, x',y',u_l^k(x,y))dxdy\bigg]. 
	\end{aligned}
	\end{equation*} 
	где $K'_u(x',y', x, y, u_l^k(x',y'))$ --- ядро интегрального оператора. %транспонированного к ядру $K'_u(x,y,$ $ x',y',u^k(x,y))$.
	% Величина $\varphi_l$ зависит от $u_l^k$.
	\let\thefootnote\relax\let\thefootnote\relax\footnotetext{ Skurydina A. F. Regularized Levenberg –- Marquardt Type Method Applied to the Structural Inverse Gravity Problem ...
 %in a Multilayer Medium and its Parallel Realization 
// Bulletin of South Ural State University. 2017. V.6, N.3. pp. 5--15}

\end{frame}
\begin{frame}
	Перепишем в дискретной форме
	\begin{equation*}\label{comp_lm_meth_disc}
	u_{l,i}^{k+1}=u_{l,i}^k-\gamma\frac{1}{\varphi_{l,i}+\bar{\alpha}}w_{l,i}\bigg[ \{A'(u_l^k)^T(A(u^k)-f_\delta)\}_i\bigg],
	\end{equation*}
	где $w_{l,i}$ --- $i$-й весовой множитель, зависящий от $l$-й границы раздела \textbf{\color{red}(Акимова, Мисилов 2015)},
	\begin{equation*}
	\begin{aligned}
	\varphi_{l,i}=\bigg[ f\Delta\sigma\sum\limits_{k=1}^{N}
	\sum\limits_{m=1}^{M}
	K'_u(x'_k,y'_m, \{x, y\}_i, u_{l,i}^k) \Delta x' \Delta y'\bigg] \notag \\ \times\bigg[f\Delta\sigma\sum\limits_{k=1}^{N}
	\sum\limits_{m=1}^{M}K'_u(x_k,y_m, \{x',y'\}_i,u_l^k(x_k,y_m))\Delta x \Delta y\bigg]. 
	\end{aligned}
	\end{equation*}
	\let\thefootnote\relax\let\thefootnote\relax\footnotetext{\footnotesize Е. Н. Акимова, В. Е. Мисилов, А. Ф. Скурыдина, А. И. Третьяков. Градиентные методы решения структурных обратных задач гравиметрии и магнитометрии на суперкомпьютере “Уран” // Вычислительные методы и программирование, 2015. Т. 16, 155–164.}
\end{frame}
%\begin{frame}{\small\textbf{Приложения к обратным задачам гравиметрии и магнитометрии}}
%	Уравнение гравиметрии в декартовой системе координат с осью $z$, направленной вниз 
%	\begin{equation*}
%	\begin{aligned}
%	A(u)=\gamma\Delta\sigma \bigg\{ &\iint_{D} \frac{1}{[(x-x')^2+(y-y')^2+H^2]^{1/2}}dx'dy' \notag\\
%	- &\iint_{D} \frac{1}{[(x-x')^2+(y-y')^2+u^2(x',y')]^{1/2}}dx'dy'\bigg\}=\Delta g(x,y),
%	\end{aligned} 
%	\end{equation*}
%	Уравнение магнитометрии имеет вид
%	\begin{equation*}\begin{aligned}
%	A(u)=\Delta J  \bigg\{&\iint_{D} \frac{H}{[(x-x')^2+(y-y')^2+H^2]^{3/2}}dx'dy' \notag\\
%	- &\iint_{D} \frac{u(x',y')}{[(x-x')^2+(y-y')^2+u^2(x',y')]^{3/2}}dx'dy' \bigg\}=\Delta G(x,y),
%	\end{aligned} \end{equation*}
%\end{frame}
%\begin{frame}{Задача 1}
%	Точное решение уравнения гравиметрии, определяющее поверхность раздела сред, задается формулой %\textbf{\color{red}(Акимова, Мисилов, Скурыдина, Третьяков, 2015)}
%	\begin{center}
%		
%		\includegraphics[width=7cm, height=4 cm]{Gravy_exact.png}            %frame~2
%	\end{center}
%	
%	\begin{center}
%		Рис.2. Модельная поверхность: $D=\{0\leqslant x\leqslant 100, \,\,0\leqslant y\leqslant 110\}$, \\ $  H=5,\,\,\Delta x=\Delta y=1,\,\,\Delta\sigma=0.21$г/см$^3$.
%	\end{center}
%\end{frame}
%\begin{frame}{\small\textbf{Результаты численных расчетов в задаче гравиметрии}}
%	Число обусловленности $\mu(A'_n(u_n^k))\approx 4.8 * 10^{8}$, спектр неотрицательный, собственные значения различны. Правило выхода из процесса итераций каждого из методов
%	$$\frac{\|\hat{u}_n-\tilde{u}_n\|_{R^n}}{\|\tilde{u}_n\|_{R^n}}\leqslant 10^{-2},$$
%	где $\hat{u}_n$ --- точное решение, а $\tilde{u}_n$ --- восстановленное каждым из четырех итерационных методов. Таким образом, точность численного решения, полученного методом Ньютона, $\alpha$-процессамии их модифицированными аналогами, гарантированно не превышала $\varepsilon=10^{-2}$.
%\end{frame}
%\begin{frame}
%	При значениях параметров $\bar\alpha=\alpha=10^{-3}$, $\gamma=1$ представлены результаты численных расчетов, где
%	$$\Delta=\frac{\|A_n(\tilde{u}_n)+\alpha(\tilde{u}_n-u^0)-f_n\|_{R^n}}{\|f_n\|_{R^n}},$$
%	$N$ --- число итераций в процессе для достижения точности $10^{-2}$, $T$ --- время реализации метода. В позициях для $\Delta$, $N$, $T$ верхняя строка соответствует основным процессам, а нижняя --- их модифицированным вариантам.
%	\begin{table}
%		\centering
%		{\scriptsize Табл.1. Относительные нормы невязок, итерации и времена счета в задаче гравиметрии}
%		
%		\begin{tabular}{|p{0.25\textwidth}|p{0.1\textwidth}|p{0.1\textwidth}|p{0.1\textwidth}|p{0.1\textwidth}|}
%			\hline
%			\rule{0cm}{0.5cm}
%			Методы & ММО & МНС & ММН & РМН \\ \hline
%			\rule{0cm}{0.5cm}
%			\multirow{$\Delta$} & 0.0048 & 0.0020 & 0.0024 & 0.0023	 \\ \cline{2-5} 
%			\rule{0cm}{0.5cm}
%			&  0.0094   & 0.0019    &  0.0019   &  0.0021   \\ \hline
%			\rule{0cm}{0.5cm}
%			\multirow{$N$} & 17  &  21   &   20  &  16    \\ \cline{2-5}
%			\rule{0cm}{0.5cm}
%			&  22   &   23  &  23   &  16   \\ \hline
%			\rule{0cm}{0.5cm}
%			\multirow{$T$ (сек)}    &  20   &  11   &  14  & 16    \\ \cline{2-5}
%			\rule{0cm}{0.5cm}
%			& 25    & 8    &  8   &   8  \\ \hline
%		\end{tabular}
%	\end{table}
%\end{frame}

\begin{frame}{3.3. Описание комплекса параллельных программ}
	\begin{itemize}
		\item на основе предложенных методов разработаны параллельные алгоритмы для вычислений на многоядерных процессорах и графических ускорителей;
		\item используются инструменты: OpenMP, CUDA;
		\item библиотеки Intel MKL, Cublas;
		\item при больших размерах сеток вычисления производились <<на лету>>: необходимый элемент матрицы вычисляется в момент умножения его на элемент вектора.
	\end{itemize}
	
	Для оценки производительности параллельных алгоритмов используются показатели {\color{red}{\textbf{ускорения и эффективности}}}
	$$
	S_m=\frac{T_1}{T_m},\quad
	\quad E_m=\frac{S_m}{m}, $$
	где $T_1$ --- время выполнения последовательного алгоритма,
	$T_m$ --- время выполнения параллельного алгоритма на $m$ ($m>1$) ядрах процессора.
\end{frame}

%\begin{frame}{Комплекс программ}
%	%\vskip -1 cm
%	\begin{figure}
%		\centering
%		\includegraphics[width=0.55\textwidth]{prog_complex}
%	\end{figure}
%%	\centering
%%	Структура комплекса
%\end{frame}

\begin{frame}{Технология OpenMP для многоядерных процессоров}
	\begin{figure}[h]
		\centering 
		\includegraphics[width=0.6\textwidth]{omp}
	\end{figure}
	\centering
	Принцип работы потоков в OpenMP.
\end{frame}
\begin{frame}{Технология CUDA для видеокарт}
	\begin{figure}[h]
		\centering
		\includegraphics[width=0.6\textwidth]{cuda}
	\end{figure}
	\centering
	Иерархия компонентов вычислительной сетки GPU.
\end{frame}
\begin{frame}{3.4. Задача 1 (сравнение методов решения задачи гравиметрии в двухслойной среде)}
	
	Точное решение уравнения гравиметрии, определяющее поверхность раздела сред, задается формулой
	\begin{equation*}
	\begin{aligned}
	\hat{u}(x,y)=5-3.21e^{-(x/10.13-6.62)^6-(y/9.59-2.93)^6}-2.78e^{-(x/9.89-4.12)^6-(y/8.63-7.435)^6}\\+3.19e^{-(x/9.89-4.82)^6-(y/8.72-4.335)^6},
	\end{aligned} 
	\end{equation*}
	\centering
		
	\includegraphics[width=\textwidth, height=0.35\textheight]{gravy_kiev2014.png}
	
	Модельная поверхность (слева) и приближенное решение (справа): $D=\{0\leqslant x\leqslant 270, \,\,0\leqslant y\leqslant 300\}$, \\ $  H=5,\,\,\Delta x=\Delta y=0.58,\,\,\Delta\sigma=0.2$ г/см$^3$.
\end{frame}

\begin{frame}
%	Критерий останова итераций 
%	$\delta=\frac{\|u_e-u_a\|}{\|u_e\|}\leqslant 0.025,$ параметр регуляризации $\alpha=\bar{\alpha}=10^{-3}$, полуширина ленты матрицы производной $\beta=1/4$, $%5\gamma=1.2$ для ПМН, $\Delta=\|A(u^k)+\alpha(u^k-u^0)-f_\delta\|/\|f_\delta\|$.\\ Размер $A'_n(u^k)$  $\approx 2.6*10^5\times2.6*10^5$.
\begin{itemize}
	\item критерий останова итераций $\delta=\frac{\|u_e-u_a\|}{\|u_e\|}\leqslant 0.025$, параметры регуляризации $\alpha=\bar{\alpha}=10^{-3}$, $\Delta=\|A(u^k)+\alpha(u^k-u^0)-f_\delta\|/\|f_\delta\|$,
	%\item параметр регуляризации $\alpha=\bar{\alpha}=10^{-3}$,
	%\item полуширина ленты матрицы производной $\beta=1/4$, $\gamma=1.2$ для ПМН,
	%\item $\Delta=\|A(u^k)+\alpha(u^k-u^0)-f_\delta\|/\|f_\delta\|$,
	\item размер $A'_n(u^k)$  $\approx 2.6*10^5\times2.6*10^5$,
	\item $T_1$ --- Intel Xeon (1 ядро), $T_8$ --- Intel Xeon (8 ядер), $T_{\textup{GPU}}$ --- NVIDIA Tesla M2050.
\end{itemize}
	
	\begin{table}[]
		\centering
		%\renewcommand{\arraystretch}{1.5}
		\scriptsize{Табл. Сравнение методов решения задачи гравиметрии на сетке $512\times512$}
		\label{table3.1}
		\begin{tabular}{|p{0.25\textwidth}|l|l|l|l|l|l|l|}
			\hline
			\multicolumn{1}{|c|}{Метод}        & \multicolumn{1}{c|}{$N$} &
			\multicolumn{1}{c|}{$\Delta$} & \multicolumn{1}{c|}{$T_1$} & \multicolumn{1}{c|}{$T_8$} &
			\multicolumn{1}{c|}{$T_{GPU}$} &
			\multicolumn{1}{c|}{$S_8$} & \multicolumn{1}{c|}{$S_{GPU}$}
			\\ \hline
			1. Метод Ньютона                      &  3        & 0.041                          &       64 мин                  &     8,82 мин &
			1 мин & 7.25 & 65 \\ \hline
			2. Модиф. метод Ньютона     &         5           & 0.042            & 55 мин                  & 7,5 мин.    &
			45 сек & 7.33 & 73  \\ \hline
			3.Метод минимальных невязок &  5               & 0.041                    & 50 мин  & 6.8 мин.  &  --   & 7.3 & ---            \\ \hline
			4. Метод Ньютона с ленточной матрицей &  4               & 0.041                    & 43 мин                  & 6,8 мин. & 30 сек   & 7.41 & 86        \\ \hline
			5. Покомпонентный метод Ньютона &  6               & 0.041                    & 20 мин  & 2,82 мин.  &  11 сек   & 7.14 & 100            \\ \hline
		\end{tabular}
	\end{table}
		\textbf{\color{blue}Вывод:} ПМН является самым экономичным по вычислительным затратам: в 3 раза быстрее методов 1--2 и в 2 раза быстрее метода 4.
\end{frame}


\begin{frame}{3.4. Задача 2 \small (сравнение методов Левенберга -- Марквардта решения задачи гравиметрии в многослойной среде}
%	\textcolor{blue}{Цель:} сравнить по точности решения и по машинным затратам регуляризованный Левенберга -- Марквардта и покомпонентный типа Левенберга -- Марквардта.
	\begin{figure}
		\centering
		\includegraphics[height=0.3\textheight]{fields}\\
		\centering\textit{Суммарное поле и поле с шумом 22\% (мГал)}
\flushleft
\vskip -5pt
Модельные поверхности построены по аналогии с границами раздела из статьи.
%		\centering
%		\includegraphics[height=0.2\textheight]{exact_hor}
%		
%		\centering\textit{Точные решения $u_0(x,y)$, $u_1(x,y)$, $u_2(x,y)$.}
	\end{figure}
%	
	\vskip -3 mm
	$H_1=8$ км, $H_2=15$ км и $H_3=30$ км. Скачки плотности $\Delta\sigma_1=0.2$ г/см$^3$, $\Delta\sigma_2=0.1$ г/см$^3$, $\Delta\sigma_3=0.1$ г/см$^3$. Шаги сетки $\Delta x=2$ км, $\Delta y=3$ км.
	
	Аддитивный гауссовский шум с $\mu=1$, $\sigma=1.15$, $\quad\textnormal{noise}=\frac{\mu+3\sigma}{f_{\textnormal{max}}}\cdot 100\%=22\%$.

	\let\thefootnote\relax\let\thefootnote\relax\footnotetext{\footnotesize П. С. Мартышко, Е.Н. Акимова, В. Е. Мисилов. О решении структурной обратной задачи гравиметрии модифицированными методами градиентного типа // Физика земли. 2016. № 5. С. 82--86.}
\end{frame}
\begin{frame}{Решения, полученные для данных без шума}
	\begin{figure}
		\centering
		\includegraphics[height=0.2\textheight]{exact_hor}

		\centering\textit{Точные решения $u_0(x,y)$, $u_1(x,y)$, $u_2(x,y)$.}
		\includegraphics[height=0.2\textheight]{levmar}

		\centering\textit{Границы, восстановленные ЛМ $\tilde{u}_0(x,y)$, $\tilde{u}_1(x,y)$, $\tilde{u}_2(x,y)$.}
		\centering
		\includegraphics[height=0.2\textheight]{clm}
	\end{figure}
	\centering\textit{Границы, восстановленные ПЛМ $\hat{u}_0(x,y)$, $\hat{u}_1(x,y)$, $\hat{u}_2(x,y)$.}

%$H_1=8$ км, $H_2=15$ км и $H_3=30$ км. Скачки плотности $\Delta\sigma_1=0.2$ г/см$^3$, $\Delta\sigma_2=0.1$ г/см$^3$, $\Delta\sigma_3=0.1$ г/см$^3$. Шаги сетки $\Delta x=2$ км, $\Delta y=3$ км.
\end{frame}
\begin{frame}{Решения, полученные для данных с шумом}
	\begin{figure}
		\centering
		\includegraphics[height=0.2\textheight]{lm_noise}
	\end{figure}
	\centering\textit{Границы, восстановленные ЛМ для данных с шумом $\tilde{u}_0(x,y)$, $\tilde{u}_1(x,y)$, $\tilde{u}_2(x,y)$.}
	\begin{figure}
		\centering
		\includegraphics[height=0.2\textheight]{clm_noise}
	\end{figure}
	\centering\textit{Границы, восстановленные ПЛМ для данных с шумом $\hat{u}_0(x,y)$, $\hat{u}_1(x,y)$, $\hat{u}_2(x,y)$.}
\end{frame}
\begin{frame}
	\begin{itemize}
		\item сетка $1000\times1000$, матрица производных $10^6\times 3*10^6$,
		\item параметр регуляризации $\alpha=10^{-3}$ и демпфирующий множитель $\gamma=1$,
		\item $\varepsilon=\frac{\|A(u)+\alpha u-f_\delta\|}{\|f_\delta\|}|=0.1$,
		\item относительные погрешности $\delta_i=\|u_a-u_e\|/\|u_e\|$,
		\item $T_1$ --- Intel Xeon (1 ядро), $T_8$ --- Intel Xeon (8 ядер),	
		\item $T_{\textup{GPU}}$ --- NVIDIA Tesla M2050.
	\end{itemize}
	\begin{table} 
		\centering
		\renewcommand{\arraystretch}{1.5} 
		{\scriptsize Табл. Результаты расчетов в задаче гравиметрии в многослойной среде}
		\small
\begin{tabular}{|c|c|c|c|c|c|c|c|}
\hline
\multirow{2}{*}{Метод} & \multirow{2}{*}{$N$} & \multirow{2}{*}{$\delta_1$} & \multirow{2}{*}{$\delta_2$} & \multirow{2}{*}{$\delta_3$} & \multirow{2}{*}{$T_1$} & \multirow{2}{*}{$T_8$} & \multirow{2}{*}{$T_{\textup{GPU}}$} \\
                       &                      &                             &                             &                             &                        &                        &                                     \\ \hline
ЛМ                     & 60                   & 0.052                       & 0.026                       & 0.051                       & 11.7 часа                & 1.4 часа                 & 35 мин.                             \\ \hline
ПЛМ                    & 20                   & 0.051                       & 0.035                       & 0.06                        & 1.2 часа                 & 10 мин.                & 3 мин.                              \\ \hline
\end{tabular}
	\end{table}

\textbf{\color{blue}Вывод:} покомпонентный метод Левенберга -- Марквардта решает задачу в 10 раз быстрее, ускорение на видеокарте в более, чем в 20 раз.
\end{frame}

%\begin{frame}{}
%	Параметр шага $\gamma=1$, параметры регуляризации $\alpha=0.1$, $\bar{\alpha}=1$, критерий останова $\Delta=\|A(u^k)+\alpha(u^k-u^0)-f_\delta\|/\|f_\delta\|<0.15$.
%	\begin{table}[H]
%		\centering
%		{\scriptsize Табл.8. Относительные ошибки для задачи с шумом}
%		\begin{tabular}{|l|c|l|l|l|}
%			\hline
%			\textbf{Метод} & \textbf{$N$} & \textbf{$\delta_1$} & \textbf{$\delta_2$} & \textbf{$\delta_3$} \\ \hline
%			ЛМ                                                    & 24                            & 0.048                                & 0.035                                & 0.059                                \\ \hline
%			ПЛМ                                                   & 8                             & 0.048                                & 0.040                                & 0.068                                \\ \hline
%		\end{tabular}
%	\end{table}
%	\textbf{\color{blue}Вывод:} решения, полученные двумя методами, близки к точному решению. Покомпонентный метод Левенберга -- Марквардта решает задачу в 10 раз быстрее.
%	использование
%	покомпонентного метода типа Левенберга – Марквардта позволяет избежать трудностей, возникающих при применении классического метода Левенберга – Марквардта: обращение плохо обусловленных матриц, высокая вычислительная сложность и большие затраты памяти. Результаты численного моделирования показывают, что относительная норма невязки покомпонентного метода сходится
%	к $\Delta$ за меньшее число итераций, чем классический регуляризованный метод Левенберга – Марквардта.
%\end{frame}

\begin{frame}{Основные результаты}
	1. Для уравнения с монотонным оператором дано обоснование двухэтапного метода на основе метода Ньютона и нелинейных аналогов $\alpha$-процессов: метода минимальной ошибки, метода наискорейшего спуска, метода минимальных невязок. Доказаны теоремы сходимости и сильная фейеровость итерационных процессов при аппроксимации регуляризованного решения. Результаты обобщены на конечномерный случай для задачи с немонотонным оператором, производная которого имеет неотрицательный спектр.
	
	2. Для решения нелинейных интегральных уравнений обратных задач гравиметрии предложены экономичные покомпонентные методы 
	типа Ньютона и типа Левенберга – Марквардта. Предложена вычислительная оптимизация метода 
	Ньютона при решении задач гравиметрии и магнитометрии.
	
	3. Разработан комплекс параллельных программ для многоядерных и графических процессоров (видеокарт) решения обратных задач гравиметрии и магнитометрии на сетках большой размерности методами ньютоновского типа и покомпонентными методами. 
	  	
\end{frame}

	
\begin{frame}{Апробация работы}
	Основные результаты по материалам диссертационной работы докладывались на конференциях:
	
	
	1. XIV и XV Уральская молодежная научная школа по геофизике (Пермь, 2013 г., Екатеринбург 2014 г.);
	
	2. Международная коференция "Параллельные вычислительные технологии" (Ростов-на-Дону, 2014 г., Екатеринбург, 2015 г., Казань, 2017 г.);
	
	3. Международная конференция "Геоинформатика: теоретические и прикладные аспекты" (Киев 2014, 2015, 2016 г.)
	
	4. Международная конференция "Актуальные проблемы вычислительной и прикладной математики" (Новосибирск, 2014 г.)
	
	5. Международный научный семинар по обратным и некорректно поставленным задачам (Москва, 2015 г.)
\end{frame}

\begin{frame}{Публикации ВАК}
\scriptsize
	\begin{enumerate}
		\item Васин В.В., Скурыдина, А.Ф. Двухэтапный метод построения регуляризующих алгоритмов для нелинейных некорректных задач // Труды ИММ УрО РАН Т.23 В.1 (2017), С. 57–74.
		\item Васин В.В., Акимова Е.Н., Миниахметова А.Ф. Итерационные алгоритмы ньютоновского типа и их приложения к обратной задаче гравиметрии // Вестник ЮУрГУ. Т.6 В.3 (2013), С. 26–37.
		\item Акимова Е. Н., Мисилов В. Е., Скурыдина А. Ф. Параллельные алгоритмы решения структурной обратной задачи магнитометрии на МВС // Вестник УГАТУ. 2014. Т.18, № 4 (65). C. 206-215.
		\item Skurydina A. F. Regularized Levenberg –- Marquardt Type Method Applied to the Structural 
Inverse Gravity Problem in a Multilayer Medium and its Parallel Realization // Bulletin of South Ural State University.
 Series: Computational Mathematics and Software Engineering. 2017. V.6, N.3. pp. 5--15.
		\item Акимова, Е. Н., Мисилов, В. Е., Скурыдина, А. Ф., Третьяков, А. И. Градиентные методы решения структурных обратных задач гравиметрии и магнитометрии 	на суперкомпьютере “Уран” // Вычислительные методы и программирование, 2015. Т. 16, 155–164.
	\end{enumerate}
\end{frame}
\begin{frame}{Публикации SCOPUS}
\scriptsize
	\begin{enumerate}
		\setcounter{enumi}{5}
		\item Akimova E., Miniakhmetova A., Martyshko M. Optimization and parallelization of Newton type methods for solving structurial gravimetry and magnetometry inverse problems // EAGE Geoinformatics 2014 – 13th Intern. Conference on Geoinformatics – Theoretical and Applied Aspects. Kiev, Ukraine. 12–15 May 2014
		\item Akimova E., Skurydina A. A componentwise Newton type method for solving the structural inverse gravity problem // EAGE Geoinformatics 2015 – 14th Intern. Conference on Geoinformatics – Theoretical and Applied Aspects. Kiev, Ukraine. 11–14 May 2015.
		\item Akimova E., Skurydina A. On solving the three-dimensional structural gravity problem for the case of a multilayered medium by the componentwize Levenberg-Marquardt method // EAGE Geoinformatics 2016 – 15th Intern. Conference on Geoinformatics – Theoretical and Applied Aspects. Kiev, Ukraine. 10–13 May 2016.
	\end{enumerate}
\end{frame}
\begin{frame}{Другие публикации}
	\begin{enumerate}
		\setcounter{enumi}{8}
			\item Мисилов В.Е., Миниахметова А.Ф., Дергачев Е.А. Решение обратной задачи гравиметрии итерационными методами на суперкомпьютере «Уран» // Труды XIV Уральской молодежной научной школы по геофизике. Пермь:
			ГИ УрО РАН.  2013. С. 187–190
			\item Миниахметова А.Ф. Сравнение быстрых методов решения структурной обратной задачи магнитометрии // Труды XV Уральской молодежной научной школы по геофизике. Екатеринбург: ИГФ УрО РАН. 2014. С.~160–162
			\item E. N. Akimova, A. F. Miniakhmetova, V. E. Misilov. Fast stable parallel algorithms for solving gravimetry and magnetometry inverse problems // The International conference "Advanced mathematics, computations and applications – 2014". Institute of Computational Mathematics and Mathematical Geophysics of Siberian Branch of RAS, Novosibirsk, Russia. June 8–11, 2014.
	\end{enumerate}
\end{frame}
\begin{frame}{Другие публикации}
	\begin{enumerate}
		\setcounter{enumi}{10}
		\item Акимова Е.Н., Мисилов В.Е., Миниахметова А.Ф. Параллельные алгоритмы решения структурной обратной задачи магнитометрии на многопроцессорных вычислительных системах  // Труды межд. конференции «Параллельные вычислительные технологии (ПАВТ’2014)», Ростов-на-Дону, 31 мар. – 4 апр. 2014 г. Челябинск:  ЮУрГУ. 2014. С. 19–29.
		\item В.В. Васин, А.Ф. Скурыдина. Регуляризованные модифицированные процессы градиентного типа для нелинейных обратных задач // Международный научный семинар по обратным и некорректно поставленным задачам. Москва, 19 – 21 ноября 2015 г.
		\item Акимова, Е. Н., Мисилов, В. Е., Скурыдина, А. Ф., Третьяков, А. И. Градиентные методы решения структурных обратных задач гравиметрии и магнитометрии на суперкомпьютере “Уран” // Труды межд. конференции «Параллельные вычислительные технологии (ПАВТ’2015)», Екатеринбург, 31 мар. – 2 апр. 2015 г. Челябинск: ЮУрГУ.  2015. С. 8–18.
	\end{enumerate}
\end{frame}

\begin{frame}\label{lastpage}
	\Huge{\centerline{Спасибо за внимание!}}
\end{frame}

%\begin{frame}
%	\begin{block}{\bf Теорема ~2.1.} 
%		Пусть $A$ --- монотонный оператор, для которого выполнены условия 
%		$$\|A(u)-A(v)\|\leqslant N_1\|u-v\|, \quad \forall u, v \in U,$$ 
%		$$\|A'(u)-A'(v)\|\leqslant N_2\|u-v\|, \quad \forall u, v \in U,$$
%		известна оценка для нормы производной в точке $u^0$, т.е.
%		$$	\|A'(u^0)\| \leqslant N_0\leqslant N_1, \quad \|u^0-u_\alpha\| \leqslant r \quad
%		u^0 \in S_r(u_\alpha),$$ $$r\leqslant \alpha/N_2, \quad 0<\alpha \leqslant \bar\alpha.$$ 
%		
%		\smallskip
%		Тогда для процесса (2.1) c $\gamma=1$ имеет место линейная скорость сходимости метода при аппроксимации единственного решения $u_\alpha$ регуляризованного уравнения (1.2)
%		$$\| u^{k}-u_\alpha \| \leqslant q^kr, \quad q=(1-\frac{\alpha}{2\bar\alpha}).$$
%	\end{block}
%\end{frame}
%\begin{frame}
%	\begin{block}{\bf Теорема ~2.2.}
%		Пусть для монотонного оператора $A$ выполнены условия $$
%		\|A(u)-A(v)\|\leqslant N_1\|u-v\|,
%		\|A'(u)-A'(v)\|\leqslant N_2\|u-v\|, \quad \forall u, v \in U,$$
%		$$\|A'(u^0)\| \leqslant N_0\leqslant N_1, \quad \|u^0-u_\alpha\| \leqslant r, $$
%		\smallskip
%		$A'(u^0)$ --- самосопряженный оператор, $\|u_\alpha-u^0\|\leqslant r \quad  
%		0\leqslant\alpha\leqslant\bar\alpha,\quad\bar\alpha\geqslant 4N_1,\quad r\leqslant\alpha/8N_2.$
%		\vspace{3mm}
%		
%		Тогда для оператора
%		$$ F(u)=(A'(u)+\bar\alpha I)^{-1}(A(u)+\alpha(u-u^0)-f_\delta) $$
%		справедлива оценка снизу
%		$$<F(u), u-u_\alpha>\geqslant\frac{\alpha}{4\bar\alpha}{\|u-u_\alpha\|}^2 \quad \forall u \in S_r(u_\alpha).$$
%	\end{block}
%\end{frame} 
%\begin{frame}
%	\begin{block}{\bf Теорема ~2.3.}
%		Пусть выполнены условия теоремы 2.2. Тогда при
%		$\gamma<\frac{\alpha\bar\alpha}{2(N_1+\alpha)^2}$
%		оператор шага $T$ процесса (2.1) при
%		$$\nu=\frac{\alpha\bar\alpha}{2\gamma(N_1+\alpha)^2}-1$$
%		удовлетворяет неравенству (2.3), для итераций $u^k$ справедливо соотношение (2.4) и имеет место сходимость
%		$$\lim_{k\to\infty}\|u^k-u_\alpha\|=0.$$
%		Если параметр $\gamma$ принимает значение ${\gamma}_{opt}=\frac{\alpha\bar\alpha}{4(N_1+\alpha)^2},$ то справедлива оценка $$\|u^k-u_\alpha\|\leqslant q^k r, \quad q=\sqrt{1-\frac{{\alpha}^2}  {16(N_1+\alpha)^2}}.$$
%	\end{block}
%\end{frame}
%\begin{frame}{3.2. Сходимость нелинейных $\alpha$-процессов}
%	\begin{block}{\bf Теорема ~3.1.}
%		Пусть для монотонного оператора $A$ выполнены условия $$\|A'(u)-A'(v)\|\leqslant N_2\|u-v\|, \quad \forall u, v \in U,	$$$$\|A'(u^0)\| \leqslant N_0\leqslant N_1, \quad \|u^0-u_\alpha\| \leqslant r,$$ и $A'(u^0)$ --- самосопряженный оператор. 
%		
%		Кроме того, для ММО параметры $\alpha$, $\bar\alpha$, $r$, $N_2$, $N_0$ удовлетворяют дополнительным соотношениям:
%		$$\alpha \leqslant \bar\alpha, \quad r\leqslant \alpha/8N_2, \quad \bar\alpha \geqslant N_0.$$
%		
%		Тогда справедливы соотношения
%		$$\|F^\varkappa(u)\|^2 \leqslant \mu_\varkappa<F^\varkappa(u), u-u_\alpha>, \quad \varkappa=-1,0,1,$$ где
%		$$
%		\mu _{-1}=\frac{4(N_1+\alpha)^2}{\alpha\bar\alpha}, \quad \mu _0= \frac{(N_1+\alpha)^2(N_1+\bar\alpha)}{\alpha{\bar\alpha}^2}, $$$$\quad \mu_1= \frac{(N_1+\alpha)^2(N_1+\bar\alpha)^2}{\alpha{\bar\alpha}^3},
%		$$
%		соответственно для ММО, МНС, ММН.
%	\end{block}
%\end{frame}
%\begin{frame}
%	\begin{block}{\bf Теорема ~3.2.}
%		Пусть выполнены условия теоремы 3.1. Тогда при
%		$$\gamma _\varkappa <\frac{2}{\mu _\varkappa}\quad (\varkappa=-1,0,1)$$
%		для последовательности $\{u^k\}$, порождаемой $\alpha$-процессом при соответствующем $\varkappa$, имеет место сходимость $$\lim_{k\to\infty}\|u^k-u_\alpha\|=0, $$ а при 
%		$\gamma{_\varkappa^{opt}}=\frac{1}{\mu_\varkappa}$
%		справедлива оценка $\|u^k-u_\alpha\|\leqslant q{_\varkappa^k}r,$ где
%		$$
%		q_{-1}=\sqrt{1-\frac{\alpha^2}{16(N_1+\alpha)^2}}, \quad q_0=\sqrt{1-\frac{\alpha^2\bar\alpha^2}{(N_1+\alpha)^2(N_1+\bar\alpha)^2}}, \quad $$$$q_1=\sqrt{1-\frac{\alpha^2\bar\alpha^4}{(N_1+\bar\alpha)^4}}.
%		$$
%	\end{block}
%\end{frame}
%\begin{frame}{\small\textbf{Случай монотонного оператора уравнения}}
%	\begin{block}{Теорема 3.5}
%		Пусть выполнены условия $$\|A(u)-A(v)\|\leqslant N_1\|u-v\|,  \quad \forall u, v \in U,$$ 
%		%$$\|A'(u)-A'(v)\|\leqslant N_2\|u-v\|, \quad \forall u, v \in U,$$
%		$$\|A'(u^0)\| \leqslant N_0, \quad \|u^0-u_\alpha\| \leqslant r,$$ 
%		
%		\smallskip
%		Оператор $A$ монотонный, $A'(u^0)$ --- самосопряженный оператор.
%		
%		\smallskip
%		Тогда при $$\gamma=\frac{2\alpha^3}{(N_0+\alpha)(N_1+\alpha)^2}$$ каждая из последовательностей, порождаемых процессами (3.3)-(3.5) сходится к регуляризованному решению $u_\alpha$ и удовлетворяет свойству Фейера.
%		Для $$\gamma=\frac{\alpha^3}{(N_0+\alpha)(N_1+\alpha)^2}$$ справедлива оценка
%		$$\|u^k-u_\alpha\|\leqslant q^k r,$$ где
%		$$q=\sqrt{1-\frac{\alpha^4}{(N_0+\alpha)^2(N_1+\alpha)^2}}.$$
%	\end{block}
%\end{frame}
%\begin{frame}{4. Оценка погрешности двухэтапного метода}
%	Согласно \textbf{\color{red}(Tautenhahn, 2002)}, при условии монотонности оператора и истокообразной представимости решения $\hat{u}$ уравнения (1.1)
%	$$u^0-\hat{u}=A'(\hat{u})v,\eqno (4.1)$$
%	справедлива оценка погрешности регуляризованного решения
%	$$\|u_\alpha^{\delta}-\hat{u}\|\leqslant\|u_\alpha^{\delta}-u_\alpha\|+\|u_\alpha-\hat{u}\|\leqslant\frac{\delta}{\alpha}+k_0\alpha,
%	\eqno (4.2)$$
%	где $k_0=(1+N_2\|v\|/2)\|v\|$, $u_\alpha^{\delta}$, $u_\alpha$ --- решения уравнения (1.2) с возмущенной $f_\delta$ и точной $f$ правой частью уравнения (1.1) соответственно. 
%	
%	Минимизируя правую часть соотношения (4.2) по $\alpha$, имеем $\alpha(\delta)=\sqrt{\delta /k_0}$, что дает оценку
%	$$\|u_{\alpha(\delta)}^{\delta, k}-\hat{u}\|\leqslant 2\sqrt{\delta k_0} \eqno (4.3)$$
%	Для итерационных процессов РМН, ММО, МНС, ММН получены оценки вида 
%	$$\|u_{\alpha(\delta)}^{\delta, k}-u_{\alpha(\delta)}^{\delta}\|\leqslant q^{k}(\delta)r \eqno (4.4)$$
%\end{frame}
%\begin{frame}{4.1. Оценка погрешности в случае монотонного оператора}
%	Объединяя оценки (4.3), (4.4), приходим к следующему утверждению
%	\begin{block}{\bf Теорема~4.1.}
%		Пусть для решения $\hat{u}$ уравнения (1.1) с монотонным оператором справедливо условие (4.1) и для метода (3.1) выполнены условия теоремы (3.1). Тогда при выборе числа итераций по правилу
%		$$ k(\delta)=\left[\frac{\ln(2\sqrt{k_0\delta}/r)}{\ln q(\delta)}\right]$$
%		справедлива оптимальная по порядку оценка погрешности двухэтапного метода
%		$$ \|u_{\alpha(\delta)}^{\delta, k}-u_{\alpha(\delta)}^{\delta}\|\leqslant 4\sqrt{k_0 \delta}.$$
%	\end{block}
%	\let\thefootnote\relax\let\thefootnote\relax\footnotetext{\footnotesize В. В. Васин В.В., Скурыдина, А.Ф. Двухэтапный метод построения регуляризующих алгоритмов для нелинейных некорректных задач // Труды ИММ УрО РАН Т.23 В.1 (2017), С. 57–74.}
%\end{frame}
%\begin{frame}{4.2. Оценка погрешности в случае оператора c положительным спектром}
%	В этой ситуации для двухэтапного алгоритма можно установить оценку для невязки --- основной характеристики точности метода при решении задачи с реальными данными.
%	Пусть регуляризованное уравнение разрешимо, тогда для его решения $u_{\alpha(\delta)}^{\delta}$ справедливо соотношение
%	$$\|A(u_{\alpha(\delta)}^{\delta})-f_\delta\|=\alpha\|u_{\alpha(\delta)}^{\delta}-u^0\|.\eqno (4.5)$$
%	Пусть для некоторой связи $\alpha(\delta)$ $\|u_{\alpha(\delta)}^{\delta}-u^0\|\leqslant m <\infty$, что влечет оценку
%	$$\|A(u_{\alpha(\delta)}^{\delta})-f_\delta\|\leqslant\alpha(\delta)m \eqno (4.6)$$
%	и сходимость $$\lim_{\delta\to 0}\|A(u_{\alpha(\delta)}^{\delta})-f_\delta\|=0,$$ при $\alpha(\delta)\to 0$, $\delta\to 0$.
%	\let\thefootnote\relax\let\thefootnote\relax\footnotetext{\footnotesize В. В. Васин В.В., Скурыдина, А.Ф. Двухэтапный метод построения регуляризующих алгоритмов для нелинейных некорректных задач // Труды ИММ УрО РАН Т.23 В.1 (2017), С. 57–74.}
%\end{frame}
%\begin{frame}
%	Пусть ${u_{\alpha(\delta)}^{\delta, k}}$ --- итерационные точки, полученные одним из методов рассмотренных выше методов. Имеем
%	$$\|A(u_{\alpha(\delta)}^{\delta, k})-f_\delta\|\leqslant\|A(u_{\alpha(\delta)}^{\delta, k})-A(u_{\alpha(\delta)}^{\delta})\|+\|A(u_{\alpha(\delta)}^{\delta})-f(\delta)\|\leqslant N_1 r q^{k(\delta)}+\alpha(\delta)m.
%	\eqno (4.7)$$
%	Выбирая, например, $\alpha(\delta)=\delta^p$ и приравнивая слагаемые в правой части (4.7), получаем правило выбора числа итерации
%	$$k(\delta)=\left [\ln(m\delta^p/N)/\ln q(\delta)\right ],$$
%	при котором справедлива оценка для невязки двухэтапного метода
%	$$\|A(u_{\alpha(\delta)}^{\delta})-f_\delta\|=2m\delta^p.\eqno (4.8)$$
%	\begin{block}{Замечание~4.3} Соотношения (4.5)---(4.8) остаются справедливыми для случая, когда матрицы $A'(u^k)$ содержат набор малых отрицательных собственных значений с тем лишь отличием, что в неравенстве (4.7) параметр $q$ во всех методах теперь вычисляется по формулам из раздела 3, в которых параметр $\bar\alpha$ заменен на $\alpha^*$ (см. замечание 3.2).
%	\end{block}
%	\let\thefootnote\relax\let\thefootnote\relax\footnotetext{\footnotesize В. В. Васин В.В., Скурыдина, А.Ф. Двухэтапный метод построения регуляризующих алгоритмов для нелинейных некорректных задач // Труды ИММ УрО РАН Т.23 В.1 (2017), С. 57–74.}
%\end{frame}
%\begin{frame}{1.2. Нелинейные аналоги альфа-процессов \textbf{(new)}}
%	Альфа-процессы для линейного  самосопряженного положительно определенного оператора  были предложены {\color{red}М.~А.~Красносельским и др. (1969)}. \\
%	Для нелинейного оператора итерационный процесс запишем в виде:
%	$$u^{k+1}=u^k-\beta_k(A(u^k)-f_{\delta}).$$ \\
%	Выбирая параметр $\beta_k$ из условия
%	$$\min_{\beta_k}{\|u^k-\beta_k(A(u^k)-f_{\delta})-z\|^2},$$ где $z$ --- решение уравнения $A'(u^k)z=y^k$, $y^k=f_{\delta}+A'(u^k)u^k-A(u^k)$ (используем разложение Тейлора в точке $u^k$), получим {\color{blue} регуляризованный метод минимальной ошибки (ММО)}
%	$$u^{k+1} =u^k - \frac{\langle B^{-1}(u^k)S_\alpha(u^k), S_\alpha (u^k)\rangle}{\|S_\alpha(u^k)\|^2}S_\alpha(u^k),$$ где $B(u^k)=A'(u^k)+\bar{\alpha}I, \quad S_\alpha(u^k)=A(u^k)+\alpha(u^k-u^0)-f_\delta$.
%\end{frame}
%\begin{frame}
%	Если использовать экстремальные принципы
%	$$\min_{\beta_k}\{\langle A'(u^k)u^{k+1},u^{k+1}\rangle-2\langle u^{k+1},y^k\rangle\},$$
%	либо
%	$$\min_{\beta_k}\{\|A'(u^k)(u^k-\beta_k(A(u^k)-f_{\delta})-y^k\|^2\},$$
%	то получаем нелинейные {\color{blue}регуляризованные методы наискорейшего спуска (МНС)}
%	$$u^{k+1} =u^k - \frac{\langle S_\alpha(u^k), S_\alpha (u^k)\rangle}{\langle B(u^k)S_\alpha(u^k), S_\alpha(u^k)\rangle}(A(u^k)+\alpha(u^k-u^0)-f_\delta)$$
%	и {\color{blue}минимальных невязок (ММН)}
%	$$u^{k+1} =u^k - \frac{\langle B(u^k)S_\alpha(u^k), S_\alpha (u^k)\rangle}{\|B(u^k)S_\alpha(u^k)\|^2}(A(u^k)+\alpha(u^k-u^0)-f_\delta).$$
%	
%	\smallskip
%	В общем виде итерационную последовательность обозначим
%	$$ u^{k+1}=u^k-\gamma\frac{\langle(A'(u^k)+\bar\alpha I)^{\varkappa}S_\alpha(u^k), S_\alpha(u^k)\rangle}{\langle(A'(u^k)+\bar\alpha I)^{\varkappa+1}S_\alpha(u^k), S_\alpha(u^k)\rangle}S_\alpha(u^k)\equiv{T(u^k)}\eqno (1.6)$$
%	при соответствующем $\varkappa=-1,0,1$, $\gamma$ --- демпфирующий множитель.
%\end{frame}
\end{document}